%%% Ne pas modifier jusqu'à la ligne 25
\documentclass[a4paper,12pt]{book}
\usepackage[utf8]{inputenc}
\usepackage[french]{babel}
%%\usepackage{CJK}
\usepackage{yhmath}
\usepackage[left=2cm,right=2cm,top=3cm,bottom=2cm, headheight=1.5cm,headsep=1.5cm]{geometry}
%%\usepackage{CJKutf8}
\usepackage{amsfonts}
\usepackage{mathrsfs}
\usepackage{amsmath,amsfonts,amssymb,dsfont}
\usepackage{graphicx}
\usepackage{subfigure}
\usepackage{enumitem}		%\enumerate-resume
\usepackage[colorlinks=true,unicode={true},hyperindex=false, linkcolor=blue, urlcolor=blue]{hyperref}
\newcommand{\myref}[1]{\ref{#1} page \pageref{#1}}

\addto\captionsfrench{\def\tablename{Tableau}}  %légendes des tableaux
\renewcommand\thesection{\Roman{section}~-~} 
\renewcommand\thesubsection{\Roman{section}.\Alph{subsection}~-~} 
\renewcommand\thesubsubsection{\Roman{section}.\Alph{subsection}.\arabic{subsubsection}~-~} 

\newcommand{\conclusion}[1]{\newline \centerline{\fbox{#1}}}

\setcounter{secnumdepth}{3}
\parindent=0pt

\usepackage{fancyhdr}
\pagestyle{fancy}

\lhead{SJTU-ParisTech} 
%%%%%%%%%%%%%%%%%%%%%%%%%%%%%%%%%%
\chead{DM3}
\rhead{Daniel 518261910024}

\begin{document}
\renewcommand{\labelitemi}{$\blacktriangleright$}
\renewcommand{\labelitemii}{$\bullet$}


\section{le « paradoxe » des anniversaires}
\subsection{}
\subsubsection{}
Lorsque l'on tire successivement au hasard n jetons avec remise
parmi M jetons numérotés de 1 à M, on peut modélise $\boxed{\Omega=\{1,2,\cdots,M\}^n}$, 

muni de \fbox{la probabilité uniforme $\mathbb{P}$} car c'est une situation d’équiprobabilité.
\subsubsection{}
$A_n \subset \Omega$ signifie que «$A_n$ contient $n$ éléments tous distincts»

C'est-à-dire \fbox{$A_n$ contient des n-arrangements de $\{1,2,\cdots,M\}$}
\subsubsection{}
Il y a $M^n$ n-uplets de $\{1,2,\cdots,M\}$ qui contient $M$ éléments, 
et $\frac{M!}{(M-n)!}$ n-arrangements lorsque $n \leq M$, 
ou $0$ n-arrangements de $\Omega$ lorsque $n > M$. Comme $\mathbb{P}$ est la possibilité uniforme, 

Finalement, 
\begin{equation}  \nonumber
    \mathbb{P}(A_n)=\frac{\#A_n}{\#\Omega}=\left\{  
                 \begin{array}{ccc}  
                 \frac{M!}{(M-n)!M^n} & & n \leq M\\  
                 0 & & n > M   
                 \end{array}  
    \right.  
\end{equation}
En fait, si l'on tire successivement au moins $M+1$ fois, la répétition est inévitable.

\subsection{}
\subsubsection{}
On modélise la situation par $\Omega=\{1,\cdots,365\}^n$, $\mathbb{P}$ la possibilité uniform sur $\Omega$.


Soit $A_n$ est l'événement que parmi une classe de $n$ élèves, au moins deux étudiants soient nés le même jour.
Alors $\overline{A_n}$ est l'événement que parmi une classe de $n$ élèves, tous les étudiants soient nés les jours tous différents, 
c'est-à-dire $\overline{A_n}$ est une élément de $\Omega$ sans répétition des éléments.

Pour les $n$ élèves, le premier peut choisir $365$ dates comme sa naissance, il en rest $364$ dates possibles pour le deuxième élève à choisir 
sans répétition $\cdots$. On notice que par la principe des tiroirs, la répétition est inévitable pour $n>365$

Finalement,  
\begin{equation}  \nonumber
    \mathbb{P}(A_n)=1-\mathbb{P}(\overline{A_n})=\left\{  
                 \begin{array}{ccc}  
                    1-\frac{365!}{(365-n)!365^n} & & n \leq 365\\  
                 0 & & n > 365  
                 \end{array}  
    \right.  
\end{equation}
\subsubsection{}
Il faut que $1-\frac{365!}{(365-n)!365^n}\geq 50\%$ avec $n \in \mathbb{N}^*$, ce qui nous donne que $\boxed{n \geq 23}$.
Le résultat me parait très surprenant 
\section{familles sommables et probabilités}
\subsection{}
On pose $\Omega=\mathbb{N}^2$, on a donc $\forall (n,p) \in \Omega$, 
$g(n,p)=\alpha \beta (1-\alpha)^n (1-\beta)^p \geq 0$ avec $(\alpha,\beta) \in (0,1)^2$. 
Et puis, 
\begin{align*}
    \sum_{(n,p) \in \Omega}g(n,p)&=\sum_{n=0}^{+\infty}\sum_{p=0}^{+\infty}\alpha \beta (1-\alpha)^n (1-\beta)^p\\
                                 &=\alpha \beta \sum_{n=0}^{+\infty}(1-\alpha)^n\sum_{p=0}^{+\infty}(1-\beta)^p
\end{align*}
Puisque $(\alpha,\beta) \in (0,1)^2$, $\sum_{n=0}^{+\infty}(1-\alpha)^n$ converge, et vaut $\frac{1}{\alpha}$. De même, 
$\sum_{p=0}^{+\infty}(1-\beta)^p$ converge, et vaut $\frac{1}{\beta}$. 

On a donc $\sum_{(n,p) \in \Omega}g(n,p)=1$, car $g$ est positif, pour tout $(n, p) \in \mathbb{N}$, 
si on pose $\mathbb{P}({(n, p)}) = g(n, p)$, $\mathbb{P}$ est bien une probabilité sur l’espace probabilisable
discret $(\mathbb{N}^2,\mathscr{P}(\mathbb{N}^2))$
\subsection{}
$(\mathbb{N},\mathscr{P}(\mathbb{N}))$ est un espace probabilisable, et la variable aléatoire $X$ définie par
\begin{equation}  \nonumber
    X=\left.  
                 \begin{array}{ccc}  
                 \mathbb{N}^2 & \to & \mathbb{N}\\  
                 (n,p) & \mapsto & n
                 \end{array}  
    \right.  
\end{equation}
$X$ est une variable aléatoire discrète, car $X(\mathbb{N}^2)=\mathbb{N}$ est dénombrable. Pour tout $k \in\mathbb{N}$, $(X=k)=\{(n,p) \in \mathbb{N}^2, X(n,p)=k\}=\{(k,p) \in \mathbb{N}^2, p \in \mathbb{N}\}$. 
Et on a des évènement $(X=i)$ et $(X=j)$ sont disjoints lorsque $i \neq j$

Alors
\begin{align*}
    \mathbb{P}(X=k)&=\sum_{p \in \mathbb{N}}\alpha \beta (1-\alpha)^k(1-\beta)^p\\
                   &=\alpha \beta (1-\alpha)^k \sum_{p=0}^{+\infty}(1-\beta)^p\\
                   &=\alpha(1-\alpha)^k
\end{align*} 

Comme $\sum_{k \in \mathbb{N}}\mathbb{P}(X=k)=\sum_{k=0}^{+\infty}\alpha(1-\alpha)^k=1$ pour $0<\alpha <1, 0<\beta <1$, 

la loi de $X$ est bien donnée par $\boxed{\mathbb{P}(X=k)=\alpha(1-\alpha)^k, \forall k \in \mathbb{N}}$.

De même, la loi de $Y$ est donnée par $\boxed{\mathbb{P}(Y=m)=\beta (1-\beta)^m, \forall m \in \mathbb{N}}$.
\subsection{}
\subsubsection{}
L'évènement $(X=k, Y=m)=\{(n,p) \in \mathbb{N}^2, X(n,p)=k, Y(n,p)=m\}=\{(k,m)\}$, donc
\begin{align*}
    \mathbb{P}(X=k,Y=m)&=\mathbb{P}(\{k,m\})\\
                       &=\alpha \beta (1-\alpha)^k (1-\beta)^m\\
                       &=[\alpha  (1-\alpha)^k] [\beta(1-\beta)^m]\\
                       &=\mathbb{P}(X=k)\mathbb{P}(Y=m)
\end{align*}
$X$ et $Y$ sont donc indépendantes. 

$$(X=Y)=\bigcup_{k\in \mathbb{N}}\left((X=k)\cap(Y=k)\right)$$

C'est une union disjointe, on a donc 
\begin{align*}
    \mathbb{P}(X=Y)&=\sum_{k \in \mathbb{N}}\mathbb{P}(X=k,Y=k)\\
                   &=\sum_{k =0}^{+\infty}\alpha  (1-\alpha)^k\beta(1-\beta)^k\\
                   &=\alpha \beta \sum_{k =0}^{+\infty}[(1-\alpha)(1-\beta)]^k
\end{align*}
Or $0<\alpha <1, 0<\beta <1$, on a donc $\boxed{\mathbb{P}(X=Y)=\frac{\alpha \beta}{\alpha+\beta-\alpha \beta}}$
\subsubsection{}
Or $X$ et $Y$ sont donc indépendantes, et $0<\alpha <1, 0<\beta <1$, on a
$$
(X>Y)=\bigcup_{k\in \mathbb{N}}\left((X=k)\cap\left(\bigcup_{n \in \mathbb{N}, n<k}(Y=n)\right)\right)
$$
C'est une union disjointe, on a donc 
\begin{align*}
    \mathbb{P}(X>Y)&=\sum_{k \in \mathbb{N}}\sum_{n \in \mathbb{N},n<k}\mathbb{P}(X=k,Y=n)\\
                   &=\sum_{k =1}^{+\infty}\sum_{n =0}^{k-1}\alpha  (1-\alpha)^k\beta(1-\beta)^n\\
                   &=\sum_{k =1}^{+\infty}\left(\alpha  (1-\alpha)^k\sum_{n =0}^{k-1}\beta(1-\beta)^n\right)\\
                   &=\alpha \sum_{k =1}^{+\infty}(1-\alpha)^k\beta \frac{1-(1-\beta)^{k}}{1-(1-\beta)}\\
                   &=\alpha \sum_{k =1}^{+\infty}(1-\alpha)^k-\alpha \sum_{k =1}^{+\infty}(1-\alpha)^k(1-\beta)^k\\
                   &=\alpha \frac{1-\alpha}{1-(1-\alpha)}-\alpha \frac{(1-\alpha)(1-\beta)}{1-(1-\alpha)(1-\beta)}\\
                   &=1-\alpha-\alpha\frac{\alpha\beta-\alpha-\beta+1}{\alpha+\beta-\alpha\beta}
\end{align*}
Finalement, on a $\boxed{\mathbb{P}(X<Y)=\frac{\beta(1-\alpha)}{\alpha+\beta-\alpha\beta}}$
\end{document}