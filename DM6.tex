%%% Ne pas modifier jusqu'à la ligne 25
\documentclass[a4paper,12pt]{book}
\usepackage[utf8]{inputenc}
\usepackage[french]{babel}
%%\usepackage{CJK}
\usepackage{yhmath}
\usepackage[left=2cm,right=2cm,top=3cm,bottom=2cm, headheight=1.5cm,headsep=1.5cm]{geometry}
%%\usepackage{CJKutf8}
\usepackage{amsfonts}
\usepackage{amsmath,amsfonts,amssymb,dsfont}
\usepackage{graphicx}
\usepackage{enumitem}		%\enumerate-resume
\usepackage[colorlinks=true,unicode={true},hyperindex=false, linkcolor=blue, urlcolor=blue]{hyperref}
\newcommand{\myref}[1]{\ref{#1} page \pageref{#1}}

\addto\captionsfrench{\def\tablename{Tableau}}  %légendes des tableaux
\renewcommand\thesection{\Roman{section}~-~} 
\renewcommand\thesubsection{\Roman{section}.\Alph{subsection}~-~} 
\renewcommand\thesubsubsection{\Roman{section}.\Alph{subsection}.\arabic{subsubsection}~-~} 

\newcommand{\conclusion}[1]{\newline \centerline{\fbox{#1}}}

\setcounter{secnumdepth}{3}
\parindent=0pt

\usepackage{fancyhdr}
\pagestyle{fancy}

\lhead{SJTU-ParisTech} 
%%%%%%%%%%%%%%%%%%%%%%%%%%%%%%%%%%
\chead{DM6}
\rhead{Daniel 518261910024}

\begin{document}
\renewcommand{\labelitemi}{$\blacktriangleright$}
\renewcommand{\labelitemii}{$\bullet$}


\section{Étude de l’équilibre}
\subsection{}

\begin{table}[h]
\begin{center}
    \begin{tabular}{l|ccccc}
    \hline
                      & $C_2H_4{(g)}$      & + & $H_2O_{(g)}$       & = & $C_2H_5OH_{(g)}$  \\ \hline
        $n_{initial}$ & $n_{1,i}=200$       &   & $n_{2,i}=200$      &   & $n_{3,i}=0$ \\ 
        $n_{final}$      & $n_{1,f}=200-\xi_{eq}$  &   & $n_{2,f}=200-\xi_{eq}$  &   & $n_{3,f}=\xi_{eq}$ \\ 
    \end{tabular}
\end{center}
\end{table}
à $T_1=298\,K$
\begin{itemize}
    \item $\Delta_rH^\circ(T_1)=\Delta_fH_{C_2H_5OH}^\circ(T_1)-\Delta_fH_{C_2H_4}^\circ(T_1)-\Delta_fH_{H_2O}^\circ(T_1)$. 
    
    \hspace*{\fill} 

    A.N. $\boxed{\Delta_rH^\circ(T_1)=-235.1-52.3+241.8=-45.6\,kJ\cdot mol^{-1}}$

    \hspace*{\fill} 

    \item $\Delta_rS^\circ(T_1)=S_{m,C_2H_5OH}^\circ(T_1)-S_{m,C_2H_4}^\circ(T_1)-S_{m,H_2O}^\circ(T_1)$. 
  
    \hspace*{\fill} 
  
    A.N. $\boxed{\Delta_rS^\circ(T_1)=282.7-188.7-219.5=-125.5\,J\cdot K^{-1}\cdot mol^{-1}}$

    \hspace*{\fill} 

    \item $\Delta_rG^\circ(T_1)=\Delta_rH^\circ(T_1)-T_1\Delta_rS^\circ(T_1)$
    
    \hspace*{\fill} 

    A.N. $\boxed{\Delta_rG^\circ(T_1)=-45.6*10^3-298*(-125.5)=-8.20\,kJ\cdot mol^{-1}}$
\end{itemize}
\subsection{}
\begin{itemize}
    \item $\sum_{i,gaz}\nu_i=-1<0$, donc selon la loi de modération de Le Chatelier, cet équilibre évolue dans le sans direct, qui diminue la quantité totale du gaz. 
    \item Comme $\Delta_rH^\circ(T_1)<0$, c'est une réaction exothermique. Selon la loi de modération de Van't Hoff, une élévation de température provoque donc une évolution dans le sens indirect.
\end{itemize}
Donc pour la synthèse industrielle, on va introduire une pression élevé, mais un compromis pour la température 
entre la vitesse de réaction et de l’équilibre est nécessaire. On peut donc choissir une température de $300^\circ C$, qui n'est pas trop haute ni trop basse pour une réaction industrielle.

\section{Évolution de l’équilibre et rendement}

\subsection{}
Par l'approximation d'Ellingham, on a $\Delta_rG^\circ(T_2)=\Delta_rG^\circ(T_1)-(T_2-T_1)*\Delta_rS^\circ$.

\hspace*{\fill} 

A.N. $\Delta_rG^\circ(T_2)=-8.2*10^3-275*(-125.5)=2.6*10^4 \,J\cdot mol^{-1}$

\hspace*{\fill} 

On a donc $K^\circ(T_2)=\exp\left(-\frac{\Delta_rG^\circ(T_2)}{RT_2}\right)=\exp(-\left(\frac{2.6*10^4}{8.31*573}\right))=4.3*10^{-3}$

\hspace*{\fill} 

Et on a $K^\circ(T_2)=Q_r=\frac{n_{C_2H_5OH}}{n_{C_2H_4}n_{H_2O}}\frac{nP^\circ}{P}=\frac{\xi_{eq}(400-\xi_{eq})*1}{(200-\xi_{eq})^2*70}$. 

\hspace*{\fill} 

Finalement, $\boxed{\xi_{eq}= 24.2\,mol}$
\subsection{}
Si on ajoute $10,0\, mol$ d’eau au mélange obtenu à l’équilibre, à pression et température constantes, seulement $n_{H_2O}$ et 
$n$ sont changé: $n_{H2O}^{'}=210-\xi_{eq}, n^{'}=410-\xi_{eq}$. 

\hspace*{\fill} 

On a donc $Q=\frac{24.2*(410-24.2)}{(200-24.2)(210-24.2)*70}=4.1*10^{-3}$

\hspace*{\fill} 

Comme $Q$ diminue, alors $Q<K^\circ(T_2)$, il faut que la réaction évolue dans le sens \fbox{direct}

\hspace*{\fill} 

Et on a $\boxed{\Delta_rG=\Delta_r G^{\circ}(T)+RT\ln(Q)=2.6*10^{4}+8.314*573*\ln(4.1*10^{-3})=-186.2\,J\cdot mol^{-1}}$

\hspace*{\fill} 

Car $\Delta_rG<0$, on arrive à le même résultat: l'équilibre evolue dans le sens direct
\subsection{}
Si on veut augment $\eta$, on a besoin d'aumenter $\xi_{eq}$, c'est-à-dire l'évolution de cette réaction dans le sens direct, 
sans aumenter $n_{C_2H_4,i}$. Il faut que $Q_r$ diminue.

On peut donc 
\begin{itemize}
    \item Rajouter plus de $H_2O_{(g)}$: Par la dérivation logarithmique, on a 
    $$\frac{dQ_r}{Q_r}=\frac{dn}{n}-\frac{dn_{H_2O}}{n_{H2O}}=dn_{H_2O}\left(1-\frac{n}{n_{H_2O}}\right)$$
    Comme $n_{H_2O}<n$, pour que $dQ_r<0$, il faut $\boxed{dn_{H_2O}>0}$ 
    \item Enlever $C_2H_5OH$ formé: $n_{C_2H_5OH}$ et $n$ sont tous diminue, $Q_r$ diminue aussi.
\end{itemize}


\end{document}