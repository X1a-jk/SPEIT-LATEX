%%% Ne pas modifier jusqu'à la ligne 25
\documentclass[a4paper,12pt]{book}
\usepackage[utf8]{inputenc}
\usepackage[french]{babel}
%%\usepackage{CJK}
\usepackage{yhmath}
\usepackage[left=2cm,right=2cm,top=3cm,bottom=2cm, headheight=1.5cm,headsep=1.5cm]{geometry}
\usepackage{amsmath,amsfonts,amssymb,dsfont}
\usepackage{graphicx}
\usepackage{enumitem}		%\enumerate-resume
\usepackage[colorlinks=true,unicode={true},hyperindex=false, linkcolor=blue, urlcolor=blue]{hyperref}
\newcommand{\myref}[1]{\ref{#1} page \pageref{#1}}

\addto\captionsfrench{\def\tablename{Tableau}}  %légendes des tableaux
\renewcommand\thesection{\Roman{section}~-~} 
\renewcommand\thesubsection{\Roman{section}.\Alph{subsection}~-~} 
\renewcommand\thesubsubsection{\Roman{section}.\Alph{subsection}.\arabic{subsubsection}~-~} 

\newcommand{\conclusion}[1]{\newline \centerline{\fbox{#1}}}

\setcounter{secnumdepth}{3}
\parindent=0pt

\usepackage{fancyhdr}
\pagestyle{fancy}

\lhead{SJTU-ParisTech} 
%%%%%%%%%%%%%%%%%%%%%%%%%%%%%%%%%%
\chead{DM1}
\rhead{Daniel 518261910024}

\begin{document}
\renewcommand{\labelitemi}{$\blacktriangleright$}
\renewcommand{\labelitemii}{$\bullet$}


\section{Question 2-2}
En un point $M$, deux ondes de même pulsation $\omega$ sont perceptibles. 
Les signaux associées sont 
$$
s_1(M,t)=a_1(M)\cos(\omega t-\phi_1(M))
$$
$$
s_2(M,t)=a_2(M)\cos(\omega t-\phi_2(M)
$$
avec les amplitudes complexes $\underline{A_1}(M)=a_1\exp(-j\phi_1(M))$ 
et $\underline{A_2}(M)=a_2\exp(-j\phi_2(M))$. 



Par principe de superposition, on a $$s(M,t)=s_1(M,t)+s_2(M,t)$$
avec amplitude complexe $\underline{A}(M,t)$. 

On a donc $\underline{A}(M,t)=\underline{A_1}(M)+\underline{A_2}(M)$, 
donc l'éclairement $\mathcal{E}$ de $s(M,t)$ satisfait
\begin{align*}
    \mathcal{E}(M)=|\underline{A}(M)|^2&=|\underline{A_1}(M)+\underline{A_2}(M)|^2\\
                &=|a_1\exp(-j\phi_1(M)+a_2\exp(-j\phi_2(M)|^2\\
                &=|a_1\cos(\phi_1(M))+ja_1\sin(\phi_1(M))+a_2\cos(\phi_2(M))+ja_2\sin(\phi_2(M))|^2\\
                &=\left(a_1\cos(\phi_1(M))+a_2\cos(\phi_2(M))\right)^2+\left(a_1\sin(\phi_1(M))+a_2\sin(\phi_2(M))\right)^2\\
                &=a_1^2+a_2^2+2a_1a_2(\cos(\phi_1(M))\cos(\phi_2(M))-\sin(\phi_1(M))\sin(\phi_2(M)))\\
                &=a_1^2+a_2^2+2a_1a_2\cos(\phi_1(M)-\phi_2(M))\\
\end{align*}
En notant $\mathcal{E}_1(M)=a_1^2$ l'éclairement de $s_1(M,t)$, $\mathcal{E}_2(M)=a_2^2$ l'éclairement de $s_2(M,t)$, 
et $\Delta \phi_{2\backslash 1}(M)$ le déphasage entre les deux signaux, on a 
$$
\boxed{\mathcal{E}(M)=\mathcal{E}_1(M)+\mathcal{E}_2(M)+2\sqrt{\mathcal{E}_1(M)\mathcal{E}_2(M)}\cos(\Delta \phi_{2\backslash 1}(M))}
$$
On arrive à le même résultat.
\end{document}