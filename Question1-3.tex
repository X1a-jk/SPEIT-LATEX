%%% Ne pas modifier jusqu'à la ligne 25
\documentclass[a4paper,12pt]{book}
\usepackage[utf8]{inputenc}
\usepackage[french]{babel}
%%\usepackage{CJK}
\usepackage{yhmath}
\usepackage[left=2cm,right=2cm,top=3cm,bottom=2cm, headheight=1.5cm,headsep=1.5cm]{geometry}
\usepackage{CJKutf8}
\usepackage{amsmath,amsfonts,amssymb,dsfont}
\usepackage{graphicx}
\usepackage{enumitem}		%\enumerate-resume
\usepackage[colorlinks=true,unicode={true},hyperindex=false, linkcolor=blue, urlcolor=blue]{hyperref}
\newcommand{\myref}[1]{\ref{#1} page \pageref{#1}}

\addto\captionsfrench{\def\tablename{Tableau}}  %légendes des tableaux
\renewcommand\thesection{\Roman{section}~-~} 
\renewcommand\thesubsection{\Roman{section}.\Alph{subsection}~-~} 
\renewcommand\thesubsubsection{\Roman{section}.\Alph{subsection}.\arabic{subsubsection}~-~} 

\newcommand{\conclusion}[1]{\newline \centerline{\fbox{#1}}}

\setcounter{secnumdepth}{3}
\parindent=0pt

\usepackage{fancyhdr}
\pagestyle{fancy}

\lhead{SJTU-ParisTech} 
%%%%%%%%%%%%%%%%%%%%%%%%%%%%%%%%%%
\chead{Question 1-3}
\rhead{夏俊恺 Daniel 518261910024}

\begin{document}
\begin{CJK}{UTF8}{gbsn}
\renewcommand{\labelitemi}{$\blacktriangleright$}
\renewcommand{\labelitemii}{$\bullet$}


\section{Question 1-3}

\begin{figure}[h]
\begin{center}
\includegraphics[scale=0.7]{1_3}
\end{center}
\caption{Profil d'une lentille}
\end{figure}



Les point $A$ et $A'$ sont conjugés par rapport à cette lentille, on sait que
le chemin optique entre deux point conjugés sont indépendant du rayon considéré.
On a: $(AA')_1=(AA')_2$, donc on peut choisir le rayon 2 qui passe par le centre:
\begin{align*}
    (AA')&=\int_{\wideparen{AA'}}n(p)dl_p\\
    &=\int_{\wideparen{AA_1}}n(p)dl_p+\int_{\wideparen{A_1A_2}}n(p)dl_p+\int_{\wideparen{A_2A'}}n(p)dl_p\\
\end{align*}

Dans $AA_1, A_1A_2, A_2A'$, les milieux sont respectivement homogènes, on a donc

\begin{align*}
    (AA')&=\int_{\wideparen{AA_1}}n(p)dl_p+\int_{\wideparen{A_1A_2}}n(p)dl_p+\int_{\wideparen{A_2A'}}n(p)dl_p\\
    &=\int_{\wideparen{AA_1}}dl_p+\int_{\wideparen{A_1A_2}}ndl_p+\int_{\wideparen{A_2A'}}dl_p\\
    &=AA_1+n*A_1A_2+A_2A'\\
    &=AA'-e+n*e\\
\end{align*}

Finalement, on a $\boxed{(AA')=AA'+(n-1)e}$

\end{CJK}
\end{document}