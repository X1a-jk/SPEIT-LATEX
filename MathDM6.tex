%%% Ne pas modifier jusqu'à la ligne 25
\documentclass[a4paper,12pt]{book}
\usepackage[utf8]{inputenc}
\usepackage[french]{babel}
%%\usepackage{CJK}
\usepackage{yhmath}
\usepackage[left=2cm,right=2cm,top=3cm,bottom=2cm, headheight=1.5cm,headsep=1.5cm]{geometry}
%%\usepackage{CJKutf8}
\usepackage{amsfonts}
\usepackage{mathrsfs}
\usepackage{amsmath,amsfonts,amssymb,dsfont}
\usepackage{graphicx}
\usepackage{subfigure}
\usepackage{enumitem}		%\enumerate-resume
\usepackage[colorlinks=true,unicode={true},hyperindex=false, linkcolor=blue, urlcolor=blue]{hyperref}
\newcommand{\myref}[1]{\ref{#1} page \pageref{#1}}

\addto\captionsfrench{\def\tablename{Tableau}}  %légendes des tableaux
\renewcommand\thesection{\Roman{section}~-~} 
\renewcommand\thesubsection{\Roman{section}.\Alph{subsection}~-~} 
\renewcommand\thesubsubsection{\Roman{section}.\Alph{subsection}.\arabic{subsubsection}~-~} 

\newcommand{\conclusion}[1]{\newline \centerline{\fbox{#1}}}

\setcounter{secnumdepth}{3}
\parindent=0pt

\usepackage{fancyhdr}
\pagestyle{fancy}

\lhead{SJTU-ParisTech} 
%%%%%%%%%%%%%%%%%%%%%%%%%%%%%%%%%%
\chead{DM}
\rhead{Daniel 518261910024}

\begin{document}
\renewcommand{\labelitemi}{$\blacktriangleright$}
\renewcommand{\labelitemii}{$\bullet$}


\section{Exercice 1}
\subsection{}
On note $A=\{(i \quad i+1)| i \in \{1,2,\cdots,n-1\}\}$ l’ensemble de transpositions et 
$\langle A\rangle$ le sous-groupe engendré par A. On va montré $\langle A\rangle=\mathscr{S}_n$ par double inclusion. 
\begin{itemize}
    \item Par définition, $\langle A\rangle$ est un sous-groupe de $\mathscr{S}_n$, donc $\langle A\rangle \subset \mathscr{S}_n$
    \item $\forall s \in \mathscr{S}_n$, si $\forall i \in \{1,2,\cdots,n\}, s(i)=i$, c'est le cas de produit vide. 
    Sinon, par la décomposition, on a $s=\tau_1 \circ \tau_2 \cdots \tau_r$, où $\forall i \in \{1,2,\cdots,r\}$, $\tau_i$ 
          une transposition, et $r \leq n-1$. On va montrer que pour chaque transposition $(i \quad j)$ tel que $(i,j)\in \{1,2,\cdots,n\}^2, i<j$, elle s'écrit
          $$(i\quad j)=(j-1 \quad j)\circ (j-2 \quad j-1)\cdots\circ (i\quad i+1)\circ (i+1\quad i+2)\circ\cdots(j-1 \quad j)$$
          On note $\gamma$ la partie à droite. 
          \begin{itemize}
              \item Par récurrence, on a 
              \begin{align*}
                \gamma(j)&=(j-1 \quad j)\circ (j-2 \quad j-1)\cdots\circ (i\quad i+1)\circ (i+1\quad i+2)\circ\cdots(j-2 \quad j-1)[j-1]\\
                         &=(j-1 \quad j)\circ (j-2 \quad j-1)\cdots\circ (i\quad i+1)\circ (i+1\quad i+2)\circ\cdots(j-3 \quad j-2)[j-2]\\
                         &=(j-1 \quad j)\circ (j-2 \quad j-1)\cdots(i+1\quad i+2)\circ (i\quad i+1)[i+1]\\
                         &=(j-1 \quad j)\circ (j-2 \quad j-1)\cdots(i+1\quad i+2)[i]\\
                         &=(j-1 \quad j)\circ (j-2 \quad j-1)\cdots(i+2\quad i+3)[i]\\
                         &=i\\
                         &=(i\quad j)[j]
              \end{align*}
              \item De même, on a 
              \begin{align*}
                \gamma(i)&=(j-1 \quad j)\circ (j-2 \quad j-1)\cdots\circ (i\quad i+1)\circ (i+1\quad i+2)\circ\cdots(j-2 \quad j-1)[i]\\
                         &=(j-1 \quad j)\circ (j-2 \quad j-1)\cdots\circ (i\quad i+1)\circ (i+1\quad i+2)\circ\cdots(j-3 \quad j-2)[i]\\
                         &=(j-1 \quad j)\circ (j-2 \quad j-1)\cdots(i+1\quad i+2)\circ (i\quad i+1)[i]\\
                         &=(j-1 \quad j)\circ (j-2 \quad j-1)\cdots(i+1\quad i+2)[i+1]\\
                         &=(j-1 \quad j)\circ (j-2 \quad j-1)\cdots(i+2\quad i+3)[i+2]\\
                         &=(j-1 \quad j)[j-1]\\
                         &=j\\
                         &=(i\quad j)[i]
              \end{align*}
              \item $\forall m\in \{1,2,\cdots,n\}^2$ tel que $m>j$ ou $m<i$, m n'est pas influencé par chacune des transpositions: $\gamma(m)=(i\quad j)(m)$
              \item $\forall k\in \{1,2,\cdots,n\}^2$ tel que $i<k<j$, on a 
              \begin{align*}
                \gamma(k)&=(j-1 \quad j)\circ \cdots(k\quad k+1)\cdots\circ (i\quad i+1)\circ\cdots(j-2 \quad j-1)[k]\\
                         &=(j-1 \quad j)\circ \cdots(k\quad k+1)\cdots\circ (i\quad i+1)\circ \cdots(k\quad k+1)\circ\cdots(j-3 \quad j-2)[k]\\
                         &=(j-1 \quad j)\circ \cdots(k\quad k+1)\cdots(i\quad i+1)\circ \cdots(k\quad k+1)[k]\\
                         &=(j-1 \quad j)\circ \cdots(k\quad k+1)\cdots(i\quad i+1)\circ \cdots(k+2\quad k+3)[k+1]\\
                         &=(j-1 \quad j)\circ \cdots(k\quad k+1)[k+1]\\
                         &=k\\
                         &=(i\quad j)[k]
              \end{align*}
          \end{itemize}
          En tout cas, on a $(i\quad j)=\gamma$, ce qui montre que chaque transposition peut s'écrit comme une produit des $(i\quad i+1)$. Par la composition, 
          $\forall s \in \mathscr{S}_n$ peut aussi s'écrit comme une produit des $(i\quad i+1)$. On a donc $\mathscr{S}_n \subset \langle A\rangle$
\end{itemize}
Finalement, $\mathscr{S}_n = \langle A\rangle$, on a donc \fbox{$\mathscr{S}_n$ est engendré par $ A$}
\subsection{}
Par la même méthode, on note $B=\{(1\quad 2),(1\quad 2\cdots\quad n)\}$ et $\langle B\rangle$ le sous-groupe engendré par B. On va montré $\langle B\rangle=\mathscr{S}_n$ par double inclusion. 
\begin{itemize}
    \item Par définition, $\langle B\rangle$ est un sous-groupe de $\mathscr{S}_n$, donc $\langle B\rangle \subset \mathscr{S}_n$
    \item $\forall s \in \mathscr{S}_n$, on a montré qu'elle s'écrit comme un produit de $(i\quad i+1)$, on va montrer que $(i\quad i+1) \in \langle B\rangle$.
    On note $c=(1\quad2\cdots n)$ le n-cycle. Puisque $c^n=Id_{\{1,2,\cdots n\}}$, on va etudier $\sigma=c^{i-1}\circ(1\quad 2)\circ c^{n-i+1}$
    \begin{itemize}
        \item lorsque $i=1$, $\sigma=c^0\circ (1 \quad 2)\circ c^n=(1 \quad 2)$
        \item pour $1<i<n$, on a $1\leq i-1\leq n-2$ et $2\leq n-i+1\leq n-1$, donc $c^{i-1}\neq Id_{\{1,2,\cdots n\}}$, $c^{n-i+1}\neq Id_{\{1,2,\cdots n\}}$. 
        On a donc 
        \begin{itemize}
            \item $\sigma(i)=c^{i-1}\circ(1\quad 2)\circ c^{n-i+1}(i)=c^{i-1}\circ(1\quad 2)(1)=c^{i-1}(2)=i+1$
            \item $\sigma(i+1)=c^{i-1}\circ(1\quad 2)\circ c^{n-i+1}(i)=c^{i-1}\circ(1\quad 2)(2)=c^{i-1}(1)=i$
            \item $\forall k\in\{1,2,\cdots,n\}\backslash\{i,i+1\}$, $c^{n-i+1}(k)\notin\{1,2\}$, donc 
            
            $\sigma(k)=c^{i-1}\circ(1\quad 2)\circ c^{n-i+1}(k)=c^{i-1}\circ c^{n-i+1}(k)=c^n(k)=k$
        \end{itemize}
    \end{itemize}
    En tout cas, on a donc $\sigma=(i\quad i+1)$. Donc $(i \quad i+1)$ s'écrit comme un produit d'éléments de $\langle B \rangle$. 
    Par le résultat précédent, on a $s$ aussi. Alors $\forall s \in \mathscr{S}_n, s \in \langle B \rangle$, donc $\mathscr{S}_n \in \langle B\rangle$.
\end{itemize}
Finalement, par double inclusion, on a $\mathscr{S}_n=\langle B\rangle$, donc \fbox{$\mathscr{S}_n$ est engendré par $B$}
\subsection{}
\subsubsection{}
\begin{itemize}
    \item réflexivité: $\forall \sigma \in \mathscr{S}_n$, $\exists Id_{\{1,2,\cdots,n\}}\in \mathscr{S}_n$, tel que 
    $\sigma=Id_{\{1,2,\cdots,n\}}\circ \sigma \circ Id_{\{1,2,\cdots,n\}}$, 
    
    donc $\sigma$ est conjuguée à elle-même($Id_{\{1,2,\cdots,n\}}^{-1}=Id_{\{1,2,\cdots,n\}}$). 
    \item symétrie: on sait que $\forall \rho \in \mathscr{S}_n$, $\rho^{-1} \in \mathscr{S}_n^2$ car c'est une bijection, alors  
    $\forall (\sigma,\sigma^{'}) \in \mathscr{S}_n^2$, $\sigma \mathcal{R}\sigma^{'} \Longleftrightarrow \exists \rho \in \mathscr{S}_n, \sigma=\rho\circ \sigma^{'}\circ \rho^{-1}
    \Longleftrightarrow \exists \rho^{'}=\rho^{-1}\in \mathscr{S}_n, \sigma^{'}=\rho^{'}\circ \sigma \circ \rho^{'-1} \Longleftrightarrow \sigma^{'}\mathcal{R}\sigma$
    \item transitivité: $\forall (a,b,c) \in \mathscr{S}_n^3$, soient $a \mathcal{R} b$, $b \mathcal{R} c$, alors 
    $\exists (\rho,\rho^{'}) \in \mathscr{S}_n^2$ tels que $a=\rho\circ b\circ \rho^{-1}$, $b=\rho^{'}\circ c\circ \rho^{'-1}$. 
    Alors $a=\rho\circ\rho^{'}\circ c\circ\rho^{'-1} \circ\rho^{-1}$. 
    
    Car $(\mathscr{S}_n,\circ)$ est un groupe, alors $\rho\circ \rho^{'} \in \mathscr{S}_n$, 
    et $(\rho\circ \rho^{'})^{-1}=\rho^{'-1}\circ \rho ^{-1} \in \mathscr{S}_n$. 
    
    En notant $\phi=\rho\circ \rho^{'}$, on a $\phi \in \mathscr{S}_n$, et $a=\phi\circ c \circ \phi^{-1}$. 
    On a donc $a\mathcal{R}c$

\end{itemize}
Alors par les définitions, on a \fbox{$\mathcal{R}$ est une relation d’équivalence}
\subsubsection{}
Dans le cours, on a montré que $\forall \sigma \in \mathscr{S}_n$, soit $(a_1\quad a_2\cdots a_p)\in \mathscr{S}_n$ un p-cycle, 
alors 

$\sigma\circ (a_1\quad a_2\cdots a_p) \circ\sigma^{-1}$ est aussi un p-cycle. 
Soin $k \in \{1,2,\cdots n\}$, $\alpha=(a_1\quad a_2\cdots a_k)$, $\beta=(b_1\quad b_2\cdots b_k)$ deux k-cycles. 

On pose $\rho \in \mathscr{S}_n$ une bijection qui envoie $b_i$ à $a_i$ $\forall i \in\{1,2,\cdots k\} $. Donc sa réciproque $\rho^{-1}$ est une bijection 
qui envoie $a_i$ à $b_i$ $\forall i \in\{1,2,\cdots k\} $. On a donc 
\begin{itemize}
    \item $\forall i \in \{1,2,\cdots,k-1\}$, on a $\alpha(a_i)=a_{i+1}$. Et on a $\rho\circ \beta \circ\rho^{-1}(a_i)=\rho\circ\beta(b_i)=\rho(b_{i+1})=a_{i+1}$
    \item $\alpha(a_k)=a_1$, et $\rho\circ \beta \circ\rho^{-1}(a_k)=\rho\circ \beta(b_k)=\rho(b_1)=a_1$
    \item $\forall j \in\{1,2,\cdots k\}\backslash\{a_1,\cdots, a_k,b_1,\cdots b_k\} $, $\alpha(j)=j$ et $\rho\circ \beta \circ\rho^{-1}(j)=\rho\circ \beta (j^{-1})=\rho(j^{-1})=j$, où
    $j^{-1}$ est l'image de $j$ par $\rho^{-1}$.(car $\rho, \rho^{-1}$) sont des bijections.
\end{itemize}
En tout cas, on a $\alpha=\rho\circ \beta \circ\rho^{-1}$, donc les deux k-cycles satisfont $\alpha \mathcal{R} \beta$. De plus, il n'y a pas de bijections entre deux cycles de cardinaux différents. 

Finalemant, \fbox{la classe d’équivalence pour $\mathcal{R}$ est l’ensemble des k-cycles de $\mathscr{S}_n$.}
\subsection{}
Dans le cours, on sait que $\forall s \in \mathscr{S}_n$, s se décompose comme un produits des k-cycles uniquement à l'ordre près de support distincts. 
\begin{itemize}
    \item sens indirect: si $(s,s^{'}) \in \mathscr{S}_n^2$ qui ont la même liste (dans l’ordre croissant) des longueurs de leurs cycles à supports disjoints. 
    Soit $s=a_1\circ \cdots \circ a_p$, $s^{'}=b_1\circ \cdots \circ b_p$, avec $a_i$, $b_i$ les cycles de même longueur(on peut ranger les compositions de $s$ et $s^{'}$ car ils sont de support distincts, donc ils commutent).
    
    Pour chaque couple des cycles $(a_i,b_i)_{i \in \{1,2,\cdots,p\}}$ (de même longueur), car ils sont dans une même class d'équivalence il exist donc $\rho \in \mathscr{S}_n$ telle que $a_i=\rho\circ b_i\circ\rho^{-1}$. 
    où $\rho$ la bijection qui envoie tous les supports de $a_i$ à ceux de $b_i$, $\forall i \in \{1,2,\cdots,p\}$ (car les $a_i$ sont de supports disjoints, et $a_i$ est $b_i$ sont de même longueurs, la bijection $\rho$ est bien définie)

    On a donc $s=\rho^p\circ(b_1\circ\cdots b_p)\circ\rho^{p^{-1}}$
    
    Car on a $\rho\in\mathscr{S}_n$, et donc 
    $\rho^{p}\in \mathscr{S}_n$, $\rho^{p^{-1}}\in \mathscr{S}_n$ car $(\mathscr{S}_n,\circ)$ est un groupe. On a donc $s=\rho^p\circ s^{'}\circ\rho^{p^{-1}}$, d'où $s \mathcal{R}s^{'}$
    \item sens direct: Soit $(s,s^{'})\in \mathscr{S}_n^2$, conjugué, alors $\exists \rho \in \mathscr{S}_n$ telle que $s=\rho\circ s^{'}\circ \rho^{-1}$. Soit $s^{'}$ se décompose somme 
    $s^{'}=\beta_1\circ\cdots\circ\beta_p$, $\forall i \in \{1,2,\cdots,p\}$, on pose $\alpha_i=\rho\circ\beta_i\circ\rho^{-1}$, on a $\alpha_i \mathcal{R} \beta_i$, elles sont dans la même classe d'équivalence, donc elles ont la même longueur. 
    De plus, les $\beta_i$ sont de supports disjoints, donc les $\alpha_i$ sont de supports disjoints car $\rho$ et $\rho^{-1}$ sont des bijections
    
    On a aussi $s=\rho\circ s^{'}\circ\rho^{-1}=\rho\circ\beta_1\circ\cdots\circ\beta_p\circ\rho^{-1}=(\rho\circ \beta_1\circ\rho^{-1})\circ\cdots\circ(\rho\circ \beta_p\circ\rho^{-1})=\alpha_1\circ\cdots\circ\alpha_p$
    
    les $(a_i)_{i \in \{1,2,\cdots,p\}}$ sont bien une décomposition des cycles de $s$, ils sont de support disjoints(car $(b_i)_{i \in \{1,2,\cdots,p\}}$ le sont, et $\rho$ est une bijection)

    donc $s$ et $s^{'}$ ont la même liste (dans l’ordre croissant) des longueurs de leurs cycles à supports disjoints.
\end{itemize}
Finalement, on a $\forall (s,s^{'}) \in \mathscr{S}_n^2$, 
elles sont conjuguées si et seulement si 

\fbox{elles ont
la même liste (dans l’ordre croissant) des longueurs de leurs cycles à supports disjoints}
\end{document}