%%% Ne pas modifier jusqu'à la ligne 25
\documentclass[a4paper,12pt]{book}
\usepackage[utf8]{inputenc}
\usepackage[french]{babel}
%%\usepackage{CJK}
\usepackage{yhmath}
\usepackage[left=2cm,right=2cm,top=3cm,bottom=2cm, headheight=1.5cm,headsep=1.5cm]{geometry}
%%\usepackage{CJKutf8}
\usepackage{amsfonts}
\usepackage{mathrsfs}
\usepackage{amsmath,amsfonts,amssymb,dsfont}
\usepackage{graphicx}
\usepackage{subfigure}
\usepackage{enumitem}		%\enumerate-resume
\usepackage[colorlinks=true,unicode={true},hyperindex=false, linkcolor=blue, urlcolor=blue]{hyperref}
\newcommand{\myref}[1]{\ref{#1} page \pageref{#1}}

\addto\captionsfrench{\def\tablename{Tableau}}  %légendes des tableaux
\renewcommand\thesection{\Roman{section}~-~} 
\renewcommand\thesubsection{\Roman{section}.\Alph{subsection}~-~} 
\renewcommand\thesubsubsection{\Roman{section}.\Alph{subsection}.\arabic{subsubsection}~-~} 

\newcommand{\conclusion}[1]{\newline \centerline{\fbox{#1}}}

\setcounter{secnumdepth}{3}
\parindent=0pt

\usepackage{fancyhdr}
\pagestyle{fancy}

\lhead{SJTU-ParisTech} 
%%%%%%%%%%%%%%%%%%%%%%%%%%%%%%%%%%
\chead{DM4}
\rhead{Daniel 518261910024}

\begin{document}
\renewcommand{\labelitemi}{$\blacktriangleright$}
\renewcommand{\labelitemii}{$\bullet$}



\section{Exercice 1}
Notons $B$ l’évènement correspondant à « le téléphone est défectueux » et 
$A$ l’évènement correspondant à « le test est positif ». On veut calculer 
la probabilité si le test soit efficace, c’est-à-dire $\mathbb{P}(B|A)$. D’après
les données,
$$\mathbb{P}(B) = \frac{1}{10000}\,\,\,\mathbb{P}(\overline{B}) = \frac{9999}{10000}$$

On a aussi $\mathbb{P}(A|B)=99\%$ et $\mathbb{P}(A|\overline{B})=0.1\%$, 

or $B$ et $\overline{B}$ est un système complet d'évènement($\mathbb{P}(\overline{B})>0, \mathbb{P}(B)>0$), 
on a 
$$\mathbb{P}(A)=\mathbb{P}(A|B)\mathbb{P}(B)+\mathbb{P}(A|\overline{B})\mathbb{P}(\overline{B})>0$$
donc 



\begin{align*}    
    \mathbb{P}(B|A)&=\frac{\mathbb{P}(A|B)\mathbb{P}(B)}{\mathbb{P}(A)}\\
                   &=\frac{\mathbb{P}(A|B)\mathbb{P}(B)}{\mathbb{P}(A|B)\mathbb{P}(B)+\mathbb{P}(A|\overline{B})\mathbb{P}(\overline{B})}\\
                   &=\frac{99\%*\frac{1}{10000}}{99\%*\frac{1}{10000}+0.1\%*\frac{9999}{10000}}
\end{align*}

Finalement, on a $\boxed{\mathbb{P}(B|A)=9\%}$, le résultat n'est pas satisfaisant.
\section{Exercice 2}
\subsection{}
On a clairement que $\forall (n,k) \in \mathbb{N}^2$, pour une variable aléatoire $X$, 

$$(X>n+k)=\{a \in \mathbb{N}, a>n+k\} \subset \{a \in \mathbb{N}, a>n\}=(X>n)$$

On a donc $(X>n+k) \cap (X>n)=(X>n+k)$. Alors, 
$$
\mathbb{P}(X>n+k)=\mathbb{P}((X>n+k) \cap (X>n))=\mathbb{P}(X>n+k|X>n)\mathbb{P}(X>n)
$$
Par la définition de « sans mémoire », on a donc $\boxed{\mathbb{P}(X>n+k)=\mathbb{P}(X>k)\mathbb{P}(X>n)}$
\subsection{}
\subsubsection{}
Lorsque $X$ suit une loi géométrique de paramètre $p \in ]0,1[$, 
on a $\forall k \in \mathbb{N}^{*}, \mathbb{P}(X=k)=p(1-p)^{k-1}$. 

Et on a $(X>n+k)=\bigcup_{i>n+k}(X=i)$, une union dénombrable et disjoint deux-à-deux.

Donc 
$$
\mathbb{P}(X>n+k)=\sum_{i>n+k}\mathbb{P}(X=i)=\sum_{i=n+k+1}^{+\infty}p(1-p)^{i-1}
$$
Or $p \in ]0,1[$, la somme converge, et on a $\mathbb{P}(X>n+k)=(1-p)^{n+k}$. 
Par la même méthode, on a $\mathbb{P}(X>n)=(1-p)^{n}, \mathbb{P}(X>k)=(1-p)^{k}$. 
Nous avons bien $\mathbb{P}(X>n+k)=\mathbb{P}(X>n)\mathbb{P}(X>k)$, ce qui montre que $X$ est sans mémoire.
\subsubsection{}
Si on suppose que $X$ modélise le rang du premier succès lors d’une succession d’épreuves de Bernoulli
indépendantes de paramètre $p$, l'évènement $(X>n+k)$ correspondant « on obtient le premier succès après $n+k$ fois ». 
On peut aussi obtenir le même résultat si d'abord on fait $n$ fois la succession sans succès: $(X>n)$, et ensuite, on 
fait une autre succession pour $k$ fois: $(X>k)$. Car ces deux étapes sont indépendantes, sa possibilité est calculée par 
multiplication. On a donc le même résultat: $\boxed{\mathbb{P}(X>n+k)=\mathbb{P}(X>n)\mathbb{P}(X>k)}$

\subsection{}
\subsubsection{}
Si on prend $k=0$, la formule devient $\forall n \in \mathbb{N}, \mathbb{P}(X>n)=\mathbb{P}(X>n)\mathbb{P}(X>0)$. 

Or $\forall n \in \mathbb{N}, \mathbb{P}(X>n)>0$, on a donc $\boxed{\mathbb{P}(X>0)=1}$
\subsubsection{}
On va prendre $p=\mathbb{P}(X=1)$
\begin{itemize}
    \item $p=\mathbb{P}(X=1)>0$ car c'est une possibilité.
    \item $(X>0)=(X=1) \cup (X>1)$ une union dénombrable et disjoint, donc 
          
          $1=\mathbb{P}(X>0)=\mathbb{P}(X=1)+\mathbb{P}(X>1)$, d'où $\mathbb{P}(X=1)=1-\mathbb{P}(X>1)<1$
\end{itemize}
On a donc $p \in ]0,1[$, et on a $\mathbb{P}(X>1)=1-p$

On va montrer l'énoncé par récurrence sur $n$
\begin{itemize}
    \item $n=0$: $\mathbb{P}(X>0)=(1-p)^0=1$, on l'a montrée
    \item Suppose que $\mathbb{P}(X>n)=(1-p)^n$, donc par la définition de sans mémoire, on a 
          $$\mathbb{P}(X>n+1)=\mathbb{P}(X>n)\mathbb{P}(X>1)=(1-p)^n*(1-p)=(1-p)^{n+1}$$
          l'énoncé est encore valide
\end{itemize}
On a donc $\boxed{\mathbb{P}(X>n)=(1-p)^n}$
\subsubsection{}
On a montré que $\mathbb{P}(X>0)=1$

$\forall n \in \mathbb{N}^{*}, (X>n-1)=(X=n) \cup (X>n)$ une union dénombrable et disjoint, donc 
          
$p=\mathbb{P}(X>n-1)=\mathbb{P}(X=n)+\mathbb{P}(X>n)$, d'où $\mathbb{P}(X=n)=\mathbb{P}(X>n-1)-\mathbb{P}(X>n)$
et on a montré que $\mathbb{P}(X>n-1)=(1-p)^{n-1}$, $\mathbb{P}(X>n)=(1-p)^{n}$, et donc 
$$
\mathbb{P}(X=n)=(1-p)^{n-1}-(1-p)^{n}=p*(1-p)^{n-1}
$$
En conclusion, $\forall n \in \mathbb{N}^{*}$, $\mathbb{P}(X=n)=p*(1-p)^{n-1}$

Or $p \in ]0,1[$, on a $\boxed{P \sim \mathscr{G}(p)}$
\end{document}