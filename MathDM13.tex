%%% Ne pas modifier jusqu'à la ligne 25
\documentclass[a4paper,12pt]{book}
\usepackage[utf8]{inputenc}
\usepackage[french]{babel}
%%\usepackage{CJK}
\usepackage{yhmath}
\usepackage[left=2cm,right=2cm,top=3cm,bottom=2cm, headheight=1.5cm,headsep=1.5cm]{geometry}
%%\usepackage{CJKutf8}
\usepackage{amsfonts}
\usepackage{mathrsfs}
\usepackage{amsmath,amsfonts,amssymb,dsfont}
\usepackage{graphicx}
\usepackage{subfigure}
\usepackage{enumitem}		%\enumerate-resume
\usepackage[colorlinks=true,unicode={true},hyperindex=false, linkcolor=blue, urlcolor=blue]{hyperref}
\newcommand{\myref}[1]{\ref{#1} page \pageref{#1}}

\addto\captionsfrench{\def\tablename{Tableau}}  %légendes des tableaux
\renewcommand\thesection{\Roman{section}~-~} 
\renewcommand\thesubsection{\Roman{section}.\Alph{subsection}~-~} 
\renewcommand\thesubsubsection{\Roman{section}.\Alph{subsection}.\arabic{subsubsection}~-~} 

\newcommand{\conclusion}[1]{\newline \centerline{\fbox{#1}}}

\setcounter{secnumdepth}{3}
\parindent=0pt

\usepackage{fancyhdr}
\pagestyle{fancy}

\lhead{SJTU-ParisTech} 
%%%%%%%%%%%%%%%%%%%%%%%%%%%%%%%%%%
\chead{DM}
\rhead{Daniel 518261910024}
\begin{document}

\renewcommand{\labelitemi}{$\blacktriangleright$}
\renewcommand{\labelitemii}{$\bullet$}


\section{}
\subsection{}
Soient $i \in \mathbb{N}^*$, $a \in \mathbb{Z}$, on a 
\begin{equation}  \nonumber
    H_i(a)=\left\{  
                 \begin{array}{rcl}  
                    \frac{a\times(a-1)\times \cdots \times(a-i+1)}{i!}=\frac{a!}{i!(a-i)!}=\binom{a}{i} & & a \geq 0\\
                    (-1)^i \frac{(i-1-a)\times (i-2-a)\times \cdots \times (-a)}{i!}=(-1)^i\frac{(i-1-a)!}{i!(-1-a)!}=(-1)^i\binom{i-1-a}{i} & & a<0\\
                 \end{array}  
    \right.  
\end{equation}
et on a $H_0=1$. En tous cas, on a $H_i(a) \in \mathbb{Z}$, on en déduit que $\boxed{\forall i \in \mathbb{N}, H_i(\mathbb{Z})\subset \mathbb{Z}}$
\subsection{}
soit $i \in \mathbb{N}^*$, on a 
$$
deg(H_i)=deg(\frac{a\times(a-1)\times \cdots \times(a-i+1)}{i!})=i
$$
et $deg(H_0)=0$

On a alors $\forall i \in \mathbb{N}$, $deg(H_i)=i$. D'après le cours, on a donc \fbox{$(H_i)_{i \in \mathbb{N}}$ est une base de $\mathbb{K}[X]$}
\subsection{}
Soient $(P,Q) \in (\mathbb{C}[X])^2$, $\lambda \in \mathbb{C}$, alors 
\begin{align*}
\Delta(\lambda P+Q) &=(\lambda P+Q)(X+1)-(\lambda P+Q)(X)\\
&=\lambda P(X+1)+Q(X+1)-\lambda P(X)-Q(X)\\
&=\lambda (P(X+1)-P(X))+(Q(X+1)-Q(X))\\
&=\lambda \Delta(P)+\Delta(Q)
\end{align*}
alors \fbox{$\Delta$ est une application linéaire}
\subsection{}
Soit $(n,i) \in \mathbb{N}^2$, montrer que 
\begin{equation}  \nonumber
    \Delta^n(H_i)=\left\{  
                 \begin{array}{rcl}  
                    0 & & i<n\\
                    1 & & i=n\\
                     \frac{X\times (X-1)\times \cdots \times (X-i+(n+1))}{(i-n)!} & & i>n\\
                 \end{array}  
    \right.  
\end{equation}
par récurrence sur $n$
\begin{itemize}
    \item initialisation: soit $n=0$, c'est le même que la définition de $H_i$
    \item hérédité: supposons que le résultat est vrai au rang $n$, pour le rang $n+1$
    \begin{itemize}
        \item Si $i=n+1$, alors $i>n$, $\Delta^n(H_i)=\frac{X\times (X-1)\times \cdots \times (X-i+(n+1))}{(i-n)!}=X$, 
        
        donc $\Delta^{n+1}(H_i)=\Delta(\Delta^n(H_i))=(X+1)-X=1$
        \item Si $i<n+1$, alors $i \leq n$, donc $\Delta^n(H_i)$ est un polynôme constant. On a $\Delta^{n+1}(H_i)=\Delta(\Delta^n(H_i))=0$
        
        \item Si $i>n+1$, alors $i >n$, donc 
        \begin{align*}
            \Delta^{n+1}(H_i)&=\Delta(\Delta^n(H_i))\\
            &=\frac{(X+1)\times X\times \cdots \times (X-i+(n+2))}{(i-n)!}-\frac{X\times (X-1)\times \cdots \times (X-i+(n+1))}{(i-n)!}\\
            &=\frac{X\times \cdots \times (X-i+(n+2))}{(i-n)!}((X+1)-(X-i+(n+1)))\\
            &=\frac{X\times \cdots \times (X-i+(n+2))}{(i-n-1)!}
        \end{align*}
        Le résultat est encore vrai au rang $n+1$
    \end{itemize}
\end{itemize}
On en déduit 
\begin{equation}  \nonumber
    \Delta^n(H_i)=\left\{  
                 \begin{array}{rcl}  
                    0 & & i<n\\
                     \frac{X\times (X-1)\times \cdots \times (X-i+(n+1))}{(i-n)!}=\boxed{H_{i-n}} & & i \geq n\\
                 \end{array}  
    \right.  
\end{equation}
Car la famille $(H_i)_{i \in \mathbb{N}}$ est une base de $\mathbb{C}[X]$, soit $P\in \mathbb{C}[X]$ de degré $m>=0$, 
il s'écrit comme $P=\sum_{k=0}^m a_kH_k$, avec les $a_i \in \mathbb{C}$. 

% Comme $\Delta$ est linéaire, on a donc $$
% \Delta^m(P)=\Delta^m(\sum_{k=0}^m a_kH_k)=\sum_{k=0}^m a_k\Delta^m(H_k)=a_m\Delta^m(H_m)+0=a_m
% $$
% Donc on a 
% $$
% P-\Delta^m(P)H_m=\sum_{k=0}^{m-1} a_kH_k
% $$
% Donc 
% $$
% \Delta^{m-1}(P-\Delta^m(P)H_m)=\Delta^{m-1}(P)-\Delta^m(P)\Delta^{m-1}(H_m)=\sum_{k=0}^{m-1} a_k\Delta^{m-1}(H_k)=a_{m-1}
% $$
% donc 
% $$
% a_{m-1}=\Delta^{m-1}(P)-X\Delta^m(P)
% $$
% Comme $\Delta$ est linéaire, on a donc $$
% \Delta^m(P)=\Delta^m(\sum_{k=0}^m a_kH_k)=\sum_{k=0}^m a_k\Delta^m(H_k)=a_m\Delta^m(H_m)+0=a_m
% $$

% Soit $1\leq n\leq m-1$, montrer que 
% $$
% a_{m-n}=\Delta^{m-n}(P)+\sum_{i=0}^{n-1}(-1)^{n-i}\Delta^{m-i}(P)\frac{X(X+1)\cdots(X+n-i-1)}{(n-i)!}
% $$
% par récurrence
% \begin{itemize}
%     \item initialisation: soit $n=1$, on a 
% $$
% P-\Delta^m(P)H_m=\sum_{k=0}^{m-1} a_kH_k
% $$
% Donc 
% $$
% \Delta^{m-1}(P-\Delta^m(P)H_m)=\Delta^{m-1}(P)-\Delta^m(P)\Delta^{m-1}(H_m)=\sum_{k=0}^{m-1} a_k\Delta^{m-1}(H_k)=a_{m-1}
% $$
% donc 
% $$
% a_{m-1}=\Delta^{m-1}(P)-X\Delta^m(P)
% $$
% Le resultat est vrai

% \item hérédité: supposons que le résultat est vrai au rang $n$, pour le rang $n+1$

% On a 
% \begin{align*}
% \Delta^{m-n-1}(P)&=\Delta^{m-n-1}\left(\sum_{k=0}^ma_kH_k\right)\\
% &=\sum_{k=0}^ma_k\Delta^{m-n-1}(H_k)\\
% &=\sum_{k=m-n-1}^ma_kH_{k-(m-n-1)}\\
% &=a_{m-n-1}+\sum_{k=m-n}^ma_kH_{k-(m-n-1)}\\
% &=a_{m-n-1}+\sum_{k=1}^{n+1}a_{m-n-1+k}H_k
% \end{align*}
% Donc 
% \begin{align*}
%     a_{m-n-1}&=\Delta^{m-n-1}(P)-\sum_{k=1}^{n+1}a_{m-n-1+k}H_k
% \end{align*}

% Le resultat est vrai au rang $n+1$
% Finalement, on a 
% \begin{align*}
% P&=\sum_{k=0}^ma_kH_k\\
% &=\sum_{k=0}^ma_{m-k}H_{m-k}\\
% &=\sum_{k=0}^m\left(\Delta^{m-k}(P)+\sum_{i=0}^{k-1}(-1)^{k-i}\Delta^{m-i}(P)\frac{X(X+1)\cdots(X+k-i-1)}{(k-i)!}\right)H_{m-k}\\
% &=\boxed{\Delta^m(P)H_m+\sum_{k=0}^{m-1}\left(\Delta^{k}(P)+\sum_{i=0}^{m-k-1}(-1)^{m-k-i}\Delta^{m-i}(P)\frac{X(X+1)\cdots(X+m-k-i-1)}{(m-k-i)!}\right)H_{k}}
% \end{align*}
On a $\forall n \in [\![0,m]\!]$, $\Delta^n(P)=a_n+\sum_{k=n+1}^ma_kH_k$, donc 
$\Delta^n(P)(0)=a_n+\sum_{k=n+1}^ma_kH_k(0)=a_n$
On a donc les coordonnées de $P$ dans cette base est $\boxed{(P(0),\Delta(P)(0),\cdots,\Delta^m(P)(0))}$

% \end{itemize}

\subsection{}
Supposons que $m \in \mathbb{N}, P=\sum_{i=0}^m a_iH_i$
\begin{itemize}
    
    \item sens indirect: soient $(a_i)_{i \in [\![0,m]\!]}$ entiers, soit $a \in \mathbb{Z}$, car $\forall i  \in [\![0,m]\!]$, $H_i(a) \in \mathbb{Z}$, 
    donc $\forall i  \in [\![0,m]\!]$, $a_iH_i(a) \in \mathbb{Z}$, donc $P(a) \in \mathbb{Z}$. Donc $P(\mathbb{Z}) \subset \mathbb{Z}$
    \item sens direct: supposons que $\forall a \in \mathbb{Z}$, $P(a) \in \mathbb{Z}$, Montrer que <<$\forall n \in [\![0,m]\!], (a_i)_{i \in [\![0,n]\!]}\in \mathbb{Z}$>> par récurrence
    \begin{itemize}
        \item $n=0$: $P(0)=\sum_{i=0}^m a_iH_i(0)=a_0 \in \mathbb{Z}$
        \item Supposons que $(a_i)_{i \in [\![0,n]\!]}\in \mathbb{Z}$, 
        on a 
        $$
        P-\sum_{k=0}^na_kH_k=\sum_{k=n+1}^ma_kH_k
        $$
        Donc 
        $$
        \Delta^{n+1}\left(P-\sum_{k=0}^na_kH_k\right)=\Delta^{n+1}\left(\sum_{k=n+1}^ma_kH_k\right)
        $$
        Donc $a_{n+1}=\Delta^{n+1}(P)(0) \in \mathbb{Z}$
    \end{itemize}
\end{itemize}
Finalement, on a $P(\mathbb{Z}) \in \mathbb{Z}$ si et seulement si les coordonnées sont entières

\subsection{}
On a $\Delta(X+1)=1=\Delta(X+2)$, donc \fbox{$\Delta$ n'est pas injective}

Pour $P=\sum_{i=0}^{m+1}b_iX^i \in \mathbb{C}[X]$, on va chercher $Q=\sum_{i=0}^{m+1}a_iX^i \in \mathbb{C}[X]$ tel que $Q=P(X+1)-P(X)$

On a 
\begin{align*}
    P(X+1)-P(X)&=\sum_{i=0}^{m+1}b_i(X+1)^i-\sum_{i=0}^{m+1}b_iX^i\\
    &=\sum_{i-0}^{m+1}b_i\sum_{j=0}^i\binom{i}{j}X^j-\sum_{i=0}^{m+1}b_iX^i\\
    &=\sum_{i-0}^{m+1}b_i\left(-X^i+\sum_{j=0}^i\binom{i}{j}X^j\right)\\
    &=\sum_{i-0}^{m+1}b_i\left(\sum_{j=0}^{i-1}\binom{i}{j}X^j\right)\\
    &=\sum_{i-0}^{m}b_{i+1}\left(\sum_{j=0}^i\binom{i+1}{j}X^j\right)\\
    &=\sum_{j=0}^m\left(\sum_{i=j+1}^{m+1}\binom{i}{j}b_i\right)X^j
\end{align*}
en faisant une sommation des paquets(de somme finie a termes positifs)

Alors pour tout $ P=\sum_{i=0}^{m+1}b_iX^i \in \mathbb{C}[X]$, il existe $Q=\sum_{j=0}^{m}a_jX^j \in \mathbb{C}[X]$, avec 
$\forall j \in [\![0,m]\!], a_j=\sum_{i=j+1}^{m+1}\binom{i}{j}b_i$ tel que $Q=P(X+1)-P(X)$, donc \fbox{$\Delta$ est surjective}

 \end{document}