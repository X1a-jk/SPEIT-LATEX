%%% Ne pas modifier jusqu'à la ligne 25
\documentclass[a4paper,12pt]{book}
\usepackage[utf8]{inputenc}
\usepackage[french]{babel}
%%\usepackage{CJK}
\usepackage{yhmath}
\usepackage[left=2cm,right=2cm,top=3cm,bottom=2cm, headheight=1.5cm,headsep=1.5cm]{geometry}
%%\usepackage{CJKutf8}
\usepackage{amsfonts}
\usepackage{mathrsfs}
\usepackage{amsmath,amsfonts,amssymb,dsfont}
\usepackage{graphicx}
\usepackage{subfigure}
\usepackage{enumitem}		%\enumerate-resume
\usepackage[colorlinks=true,unicode={true},hyperindex=false, linkcolor=blue, urlcolor=blue]{hyperref}
\newcommand{\myref}[1]{\ref{#1} page \pageref{#1}}

\addto\captionsfrench{\def\tablename{Tableau}}  %légendes des tableaux
\renewcommand\thesection{\Roman{section}~-~} 
\renewcommand\thesubsection{\Roman{section}.\Alph{subsection}~-~} 
\renewcommand\thesubsubsection{\Roman{section}.\Alph{subsection}.\arabic{subsubsection}~-~} 

\newcommand{\conclusion}[1]{\newline \centerline{\fbox{#1}}}

\setcounter{secnumdepth}{3}
\parindent=0pt

\usepackage{fancyhdr}
\pagestyle{fancy}

\lhead{SJTU-ParisTech} 
%%%%%%%%%%%%%%%%%%%%%%%%%%%%%%%%%%
\chead{DM}
\rhead{Daniel 518261910024}
\begin{document}

\renewcommand{\labelitemi}{$\blacktriangleright$}
\renewcommand{\labelitemii}{$\bullet$}


\section{}
\subsection{}
Pour tout $n \in \mathbb{N}$, on pose $u_n=\frac{1}{n^\alpha}z^n$. Pour $z \neq 0$, $u_n(z) \neq 0$, donc 
$$
\left|\frac{u_{n+1}}{u_n}\right|=\left|\frac{\frac{1}{(n+1)^\alpha}z^{n+1}}{\frac{1}{(n)^\alpha}z^{n}}\right|=\left|(\frac{n}{n+1})^\alpha z\right|\xrightarrow[n \to +\infty]{}|z|
$$
Par le critère de d'Alembert

$\sum u_n(z)$ est absolument convergente lorsque $|z|<1$

$\sum u_n(z)$ est grossièrement divergente lorsque $|z|>1$

On a donc $\boxed{R(\sum \frac{1}{n^\alpha}z^n)=1}$

\subsection{}
Pour tout $n \in \mathbb{N}$, on pose $u_n=n!\times z^{n^2}$. Pour $z \neq 0$, $u_n(z) \neq 0$, donc 
$$
\left|\frac{u_{n+1}}{u_n}\right|=\left|(n+1)\times z^{2n+1}\right|%\xrightarrow[n \to +\infty]{}|z|
$$
Par le critère de d'Alembert

lorsque $|z|<1$, $\left|(n+1)\times z^{2n+1}\right|\xrightarrow[n \to +\infty]{}0$, $\sum u_n(z)$ est absolument convergente

lorsque $|z|\geq 1$, $\left|(n+1)\times z^{2n+1}\right|\xrightarrow[n \to +\infty]{}+\infty$, $\sum u_n(z)$ est grossièrement divergente

On a donc $\boxed{R(\sum n!\times z^{n^2})=1}$

\subsection{}
Pour tout $n \in \mathbb{N}$, on pose $u_n=n!\times z^{n!}$. Pour $z \neq 0$, $u_n(z) \neq 0$, donc 
$$
\left|\frac{u_{n+1}}{u_n}\right|=\left|(n+1)\times z^{n \times n!}\right|%\xrightarrow[n \to +\infty]{}|z|
$$
Par le critère de d'Alembert

lorsque $|z|<1$, $\left|(n+1)\times z^{n \times n!}\right|\xrightarrow[n \to +\infty]{}0$, $\sum u_n(z)$ est absolument convergente

lorsque $|z|\geq 1$, $\left|(n+1)\times z^{n \times n!}\right|\xrightarrow[n \to +\infty]{}+\infty$, $\sum u_n(z)$ est grossièrement divergente

On a donc $\boxed{R(\sum n!\times z^{n!})=1}$

\section{}
\subsection{}
Pour tout $n \in \mathbb{N}$, on pose $u_n=\frac{z^{n+1}}{n+1}$. Pour $z \neq 0$, $u_n(z) \neq 0$, donc 
$$
\left|\frac{u_{n+1}}{u_n}\right|=\left|(\frac{n+1}{n+2})\times z\right|\xrightarrow[n \to +\infty]{}|z|
$$
Par le critère de d'Alembert

$\sum u_n(z)$ est absolument convergente lorsque $|z|<1$

$\sum u_n(z)$ est grossièrement divergente lorsque $|z|>1$

On a donc $\boxed{R(\sum \frac{z^{n+1}}{n+1})=1}$
\subsection{}
On a $x=|z| \in \mathbb{R_+}$, donc 
\begin{itemize}
    \item lorsque $x>1$, $\sum u_n(z)$ est grossièrement divergente
    \item lorsque $x<1$, on note $$
    S(x)=\sum_{n \geq 0} \frac{z^{n+1}}{n+1}=\sum_{n \geq 0}\frac{e^{(n+1)i\theta}}{n+1}x^{n+1}
    $$
    la série dérivée est donc $\sum_{n \geq 0}e^{(n+1)i\theta}x^n $, de même rayon de convergence $R=1$. 

    Pour tout $x \in [0,1[$, on a donc 
    $$
    S^{'}(x)=\sum_{n = 0}^{+ \infty}e^{(n+1)i\theta}x^n=e^{i\theta}\sum_{n = 0}^{+ \infty}(e^{i\theta}x)^n=\frac{e^{i\theta}}{1-xe^{i\theta}}
    $$
    car $|e^{i\theta}x|=|e^{i\theta}||x|<1$
    Donc $$
    S(x)=-\ln(1-e^{i\theta}x)+C
    $$
    Comme $S(0)=0$, on a $C=0$, donc $S(x)=-\ln(1-e^{i\theta}x)$
    \item lorsque $x=1$, $S(x)=\sum_{n \geq 0}\frac{e^{(n+1)i\theta}}{n+1}$. On sait que cette série converge sauf en $z=1$ d'après le cours.
\end{itemize}


\subsection{}
% Soient $\theta \in ]0,2\pi[$, $N \in \mathbb{N}$, on a donc 
% \begin{align*}
% \sum_{n=0}^{N} \frac{e^{(n+1)i\theta}}{n+1}&=\sum_{n=1}^{N+1} \frac{e^{ni\theta}}{n}\\
% &=\frac{1}{N+1}\sum_{p=0}^{N+1}e^{ip\theta}+\sum_{k=2}^{N+1}\left[\left(\frac{1}{k-1}-\frac{1}{k}\right)\times \sum_{p=0}^{k-1}e^{ip\theta}\right]
% \end{align*}
% et on a $\forall A \in \mathbb{N}$, $\left|\sum_{p=0}^{A}e^{ip\theta}\right|\leq \frac{1}{|\sin\frac{\theta}{2}|}$

% Donc on a 
% \begin{align*}
%     \sum_{n=0}^{N} \frac{e^{(n+1)i\theta}}{n+1} &\xrightarrow[N \to +\infty]{}
%     0+\frac{1}{|\sin\frac{\theta}{2}|}\sum \frac{1}{k^2}\\
%     &=\frac{1}{|\sin\frac{\theta}{2}|}
%     \end{align*}
Soient $\theta \in ]0,2\pi[$, on a $x=|z|=1$. Par les résultats précédents, on a $\sum_{n=0}^{+\infty} \frac{e^{(n+1)i\theta}}{n+1}$ converge, et 
par continuité radicale, on a 
$$
\boxed{\sum_{n=0}^{+\infty} \frac{e^{(n+1)i\theta}}{n+1}=\lim_{x\to 1^-}S(x)=-(1-e^{i\theta})}
$$


 \end{document}