%%% Ne pas modifier jusqu'à la ligne 25
\documentclass[a4paper,12pt]{book}
\usepackage[utf8]{inputenc}
\usepackage[french]{babel}
%%\usepackage{CJK}
\usepackage{yhmath}
\usepackage[left=2cm,right=2cm,top=3cm,bottom=2cm, headheight=1.5cm,headsep=1.5cm]{geometry}
%%\usepackage{CJKutf8}
\usepackage{amsfonts}
\usepackage{mathrsfs}
\usepackage{amsmath,amsfonts,amssymb,dsfont}
\usepackage{graphicx}
\usepackage{subfigure}
\usepackage{enumitem}		%\enumerate-resume
\usepackage[colorlinks=true,unicode={true},hyperindex=false, linkcolor=blue, urlcolor=blue]{hyperref}
\newcommand{\myref}[1]{\ref{#1} page \pageref{#1}}

\addto\captionsfrench{\def\tablename{Tableau}}  %légendes des tableaux
\renewcommand\thesection{\Roman{section}~-~} 
\renewcommand\thesubsection{\Roman{section}.\Alph{subsection}~-~} 
\renewcommand\thesubsubsection{\Roman{section}.\Alph{subsection}.\arabic{subsubsection}~-~} 

\newcommand{\conclusion}[1]{\newline \centerline{\fbox{#1}}}

\setcounter{secnumdepth}{3}
\parindent=0pt

\usepackage{fancyhdr}
\pagestyle{fancy}

\lhead{SJTU-ParisTech} 
%%%%%%%%%%%%%%%%%%%%%%%%%%%%%%%%%%
\chead{DM}
\rhead{Daniel 518261910024}

\begin{document}
\renewcommand{\labelitemi}{$\blacktriangleright$}
\renewcommand{\labelitemii}{$\bullet$}


\section{Exercice 1}
\subsection{}
Comme $f$ est positive, on a $\forall x \in \mathbb{R}_+, f^{"}(x)\geq f(x)\geq 0$, 
donc $f^{"}$ est positive, \fbox{$f$ est donc convexe}

Car $f$ est continue sur $\mathbb{R}_+$, on a donc la courbe représentative de $f$ est au-dessus ses tangentes. 

Soit $a \in \mathbb{R}_+$, on a donc $\forall x \in \mathbb{R}_+, f(x) \geq f^{'}(a)(x-a)+f(a)$. Si $f^{'}(a) >0$, on a 
$f(x)\geq f^{'}(x-a)+f(a)\xrightarrow[x \to +\infty]{}  +\infty$, c'est l'absurde($f$ est bornée sur $\mathbb{R}_+$ d'après l'énoncé). Donc $f^{'}(a)\leq 0$ 

On a donc $\forall a \in \mathbb{R}_+, f^{'}(a)\leq 0$, donc $f^{'}$ est négative sur $\mathbb{R}_+$, \fbox{$f$ est donc décroissante}

\subsection{}

$f$ est décroissante est bornée sur $\mathbb{R}_+$, donc $f$ admet une limite, on le note $l$

Car $f^{"}$ est positive sur $\mathbb{R}_+$, donc $f^{'}$ est croissante sur $\mathbb{R}_+$. $f^{'}$ admet aussi une limite car elle est majorée par $0$, on le note $l^{'}$

Soit $a \in \mathbb{R}_+$, soit $x\in \mathbb{R}_+$ tel que $x>a$, d'après TAF, on a il existe $\xi \in ]a,x[$, tel que $f^{'}(\xi)=\frac{f(x)-f(a)}{x-a}$, 
on a donc $l^{'}=f^{'}(\xi)\xrightarrow[x \to +\infty]{}(l-f(a))\times 0=0$. Alors $\boxed{\lim_{x \to +\infty}f^{'}(x)=l^{'}=0}$,  

On a aussi $\forall x\in \mathbb{R}_+,f^{"}(x)\geq f(x)\geq l$ car f est décroissante, donc de même méthode,  
soit $b \in \mathbb{R}_+$, soit $x\in \mathbb{R}_+$ tel que $x>b$, d'après TAF, on a il existe $\lambda \in ]b,x[$, tel que $l\leq f^{"}(\lambda)=\frac{f^{'}(x)-f^{'}(b)}{x-b}$, 
donc $l \leq \frac{l^{'}-f^{'}(b)}{x-b} \xrightarrow[x \to +\infty]{}0$. Car $f$ est positive, donc $l \geq 0$, donc $\boxed{\lim_{x \to +\infty}f(x)=l=0}$
\subsection{}
On a $\forall x\in \mathbb{R}_+$, $h^{'}(x)=[f^{"}(x)-f(x)]e^{-x}\leq 0$, \fbox{$h$ est donc croissante}

De puis, on a $h(x)=\frac{f(x)+f^{'}(x)}{e^x}\xrightarrow[x \to +\infty]{}0$, car $h$ est croissante, on a $\boxed{\forall x \in \mathbb{R}_+, h(x) \leq 0}$. 

On a donc $\forall x \in \mathbb{R}_+$, $f(x)+f^{'}(x) \leq 0$ car $x \mapsto e^{-x}$ est positive. Et on a $\forall x \in \mathbb{R}_+, g^{'}(x)=[f(x)+f^{'}(x)]e^x\leq 0$, 
\fbox{g est donc décroissante}
\subsection{}
Car g est décroissante, on a $\forall x \in \mathbb{R}_+$, $f(0)e^0 \geq f(x)e^x$, 
donc  $\boxed{\forall x \in \mathbb{R}_+, f(x) \leq f(0)e^{-x}}$


\section{Exercice 2}
\subsection{}
Soit $x \in \mathbb{R}$, on note $u_n=\frac{|x-a_n|}{3^n}$. C'est une série à termes positifs. 

On a $u_n \sim \frac{1}{3^n}$ car $(a_n)_{n \in \mathbb{N}}$ est bornée, donc $\sum u_n$ converge car $\sum \frac{1}{3^n}$ est une série géométrique 
converge. Donc \fbox{f est bien définie sur $\mathbb{R}$} 
Soit $(x,y) \in \mathbb{R}^2$, $t \in [0,1]$, on a 
\begin{align*}
    f(tx+(1-t)y)&=\sum_{n=0}^{+\infty}\frac{|tx+(1-t)y-a_n|}{3^n}\\
    &\leq\sum_{n=0}^{+\infty}\frac{|tx-ta_n|}{3^n}+\sum_{n=0}^{+\infty}\frac{|(1-t)y-(1-t)a_n|}{3^n}\\
    &=t\sum_{n=0}^{+\infty}\frac{|x-a_n|}{3^n}+(1-t)\sum_{n=0}^{+\infty}\frac{|y-a_n|}{3^n}\\
    &=tf(x)+(1-t)f(y)
\end{align*}
\fbox{f est donc convexe}
\subsection{}
    soit $h \in \mathbb{R}$, on a 
    \begin{align*}
    \left|f(a_0+h)-f(a_0)-|h|\right|&=\left|\sum_{n=0}^{+\infty}\frac{|a_0+h-a_n|}{3^{n}}-\sum_{n=0}^{+\infty}\frac{|a_0-a_n|}{3^{n}}-|h|\right|\\
    &=\left|\sum_{n=1}^{+\infty}\frac{|a_0+h-a_n|}{3^{n}}+|h|-\sum_{n=1}^{+\infty}\frac{|a_0-a_n|}{3^{n}}-|h|\right|\\
    &\leq \left|\sum_{n=1}^{+\infty}\frac{|a_0-a_n|}{3^{n}}+\sum_{n=1}^{+\infty}\frac{|h|}{3^{n}}-\sum_{n=1}^{+\infty}\frac{|a_0-a_n|}{3^{n}}\right|\\
    &=\left|\sum_{n=1}^{+\infty}\frac{|h|}{3^{n}}\right|\\
    &=|h|\left(\sum_{n=1}^{+\infty}\frac{1}{3^{n}}\right)\\
    &\leq \frac{|h|}{2}
    \end{align*}
    Car $\sum_{n=1}^{+\infty}\frac{1}{3^{n}}$ converge(série géométrique convergente), et vaut $\frac{1/3}{1-1/3}=\frac{1}{2}$

    
Donc, on a $\boxed{\forall h \in \mathbb{R}, \left|f(a_0+h)-f(a_0)-|h|\right|\leq \frac{|h|}{2}}$

Car f est convexe, elle admet les dérivées à droite et à gauche en tous les $a_i, i \in \mathbb{N}$

On le montre par l'absurde: supposons que f est dérivable en tous les points de la suite $(a_n)_{n\in \mathbb{N}}$.

Soit $i \in \mathbb{N}$, on note $f^{'}_d(a_i)$ la dérivée de $a_i$ à droite, $f^{'}_g(a_i)$ la dérivée de $a_i$ à gauche, $f^{'}_g(a_i)\leq f^{'}_d(a_i)$ car f est convexe

On suppose que 
$f^{'}_d(a_i)=f^{'}_g(a_i)$. 

On a $\left|f(a_0+h)-f(a_0)-|h|\right| \leq \frac{|h|}{2}$, soit $h \neq 0$
\begin{itemize}
    \item $h > 0$, on a donc 
    $$
    \frac{-h}{2} \leq f(a_0+h)-f(a_0)-h\leq \frac{h}{2}
    $$
    donc 
    $$
    \frac{1}{2} \leq \frac{f(a_0+h)-f(a_0)}{h}\leq \frac{3}{2}
    $$
    donc 
    $$
    \frac{1}{2} \leq f^{'}_d(a_0)=\lim_{h\to 0^+}\frac{f(a_0+h)-f(a_0)}{h}\leq \frac{3}{2}
    $$
    \item $h < 0$, on a donc 
    $$
    \frac{h}{2} \leq f(a_0+h)-f(a_0)+h\leq -\frac{h}{2}
    $$
    donc 
    $$
    -\frac{3}{2} \leq \frac{f(a_0+h)-f(a_0)}{h}\leq -\frac{1}{2}
    $$
    donc 
    $$
    -\frac{3}{2} \leq f^{'}_g(a_0)=\lim_{h\to 0^-}\frac{f(a_0+h)-f(a_0)}{h}\leq -\frac{1}{2}
    $$
\end{itemize}
on a 
$$
  f^{'}_d(a_0) \geq \frac{1}{2} > -\frac{1}{2}\geq f^{'}_g(a_0)
$$
Donc il est impossible que $f^{'}_d(a_0)=f^{'}_g(a_0)$, c'est l'absurde. 

Finalement, \fbox{f n'est pas dérivable en $a_0$}
\subsection{}
Initialisation: f n'est pas dérivable en $a_0$

Hérédité: on suppose que $(H_k):$ f n'est pas dérivable en tous les $a_i$, $i \in [\![0,\cdots,k]\!]$ soit vrai.

Pour le rang $k+1$, on a 
$$f(x)=\sum_{n=0}^{+\infty}\frac{|x-a_n|}{3^n}=\sum_{n=0}^{k}\frac{|x-a_n|}{3^n}+\sum_{n=k+1}^{+\infty}\frac{|x-a_n|}{3^n}$$
Si on note $\forall i \in \mathbb{N}, b_i=a_{i+k+1}$, donc la suite $(b_n)_{n \in \mathbb{N}}$ est aussi une suite de réels bornée. 

On a donc 
$$
\sum_{n=k+1}^{+\infty}\frac{|x-a_n|}{3^n}=\frac{1}{3^{k+1}}\sum_{n=0}^{+\infty}\frac{|x-b_n|}{3^n}
$$
qui n'est pas dérivable en $b_0=a_{k+1}$

Et on a $\sum_{n=0}^{k}\frac{|x-a_n|}{3^n}$ est dérivable en $a_{k+1}$ lorsque $a_{k+1} \neq a_i, \forall i \in [\![0,\cdots,k]\!]$
(lorsque il existes $ i \in [\![0,\cdots,k]\!]$ tel que $a_i=a_{k+1}$, f est indérivable en cet $a_i=a_{k+1}$)

Alors, f n'est pas dérivable en $a_{k+1}$. Par notre hypothèse que f n'est pas dérivable en tous les $a_i$, $i \in [\![0,\cdots,k]\!]$, 
on a donc f n'est pas dérivable en tous les $a_i$, $i \in [\![0,\cdots,k+1]\!]$, $(H_{k+1})$ est encore vraie.

Finalement, on a \fbox{f n'est pas dérivable en tous les $(a_n)_{n \in \mathbb{N}}$}
\end{document}