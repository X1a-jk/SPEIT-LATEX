%%% Ne pas modifier jusqu'à la ligne 25
\documentclass[a4paper,12pt]{book}
\usepackage[utf8]{inputenc}
\usepackage[french]{babel}
%%\usepackage{CJK}
\usepackage{yhmath}
\usepackage[left=2cm,right=2cm,top=3cm,bottom=2cm, headheight=1.5cm,headsep=1.5cm]{geometry}
\usepackage{CJKutf8}
\usepackage{amsmath,amsfonts,amssymb,dsfont}
\usepackage{graphicx}
\usepackage{mathrsfs}
\usepackage{enumitem}		%\enumerate-resume
\usepackage[colorlinks=true,unicode={true},hyperindex=false, linkcolor=blue, urlcolor=blue]{hyperref}
\newcommand{\myref}[1]{\ref{#1} page \pageref{#1}}

\addto\captionsfrench{\def\tablename{Tableau}}  %légendes des tableaux
\renewcommand\thesection{\Roman{section}~-~} 
\renewcommand\thesubsection{\Roman{section}.\Alph{subsection}~-~} 
\renewcommand\thesubsubsection{\Roman{section}.\Alph{subsection}.\arabic{subsubsection}~-~} 

\newcommand{\conclusion}[1]{\newline \centerline{\fbox{#1}}}

\setcounter{secnumdepth}{3}
\parindent=0pt

\usepackage{fancyhdr}
\pagestyle{fancy}

\lhead{SJTU-ParisTech} 
%%%%%%%%%%%%%%%%%%%%%%%%%%%%%%%%%%
\chead{TD 2-1}
\rhead{Daniel 518261910024}

\begin{document}
\begin{CJK}{UTF8}{gbsn}
\renewcommand{\labelitemi}{$\blacktriangleright$}
\renewcommand{\labelitemii}{$\bullet$}


\section{Influence de la taille finie d'un détecteur}

\begin{figure}[h]
\begin{center}
\includegraphics[scale=0.7]{TD2-1}
\end{center}
\caption{Figure de ce détecteur}
\end{figure}

\subsection{}
On a la puissance
\begin{align*}
P&=\iint_{surface} \mathcal{E}(M)\,dS \\
 &=\int_{y_c-\frac{l}{2}}^{y_c+\frac{l}{2}}\int_{x_c-\frac{L}{2}}^{x_c+\frac{L}{2}}
 \mathcal{E}_0(1+\cos(Kx))\,dx\,dy\\
 &=\mathcal{E}_0\int_{y_c-\frac{l}{2}}^{y_c+\frac{l}{2}}\,dy\int_{x_c-\frac{L}{2}}^{x_c+\frac{L}{2}}
 1+\cos(Kx)\,dx\\
 &=\mathcal{E}_0l\left[x+\frac{1}{K}\sin(Kx)\right]_{x=x_c-\frac{L}{2}}^{x=x_c+\frac{L}{2}}\\
 &=\mathcal{E}_0l\left(L+\frac{1}{K}(\sin(K(x_c+\frac{L}{2}))-\sin(K(x_c-\frac{L}{2})))\right)
\end{align*}
Car on a $\sin(\alpha)-\sin(\beta)=2\cos(\frac{\alpha+\beta}{2})\sin(\frac{\alpha-\beta}{2})$, donc
\begin{align*}
P&=\mathcal{E}_0l\left(L+\frac{1}{K}(2\cos(Kx_c)\sin(\frac{LK}{2}))\right)\\
 &=\mathcal{E}_0Ll\left(1+2\frac{\sin(\frac{LK}{2})}{LK}\cos(Kx_c)\right)\\
\end{align*}
On a donc: $\boxed{f(L)=2\frac{\sin(\frac{LK}{2})}{LK}}$

\subsection{}
Quand $L\ll\frac{1}{K}$, on a $LK\ll1$, donc $f(L)=\frac{\sin(\frac{LK}{2})}{\frac{LK}{2}}\sim1$, 
donc $P=\mathcal{E}_0Ll\left(1+f(L)\cos(Kx_c)\right) \sim \mathcal{E}_0Ll\left(1+\cos(Kx_c)\right) = 
\boxed{Ll\mathcal{E}(x_c)}$, $P$ est donc proportionnelle à l'éclairement $\mathcal{E}(x_c)$. On a aussi
$L\ll\lambda=\frac{2\pi}{K}$, la longueur caractéristique du détecteur est beaucoup plus petite que la longueur
d'onde, tous les points sur le détecteur est relativement proche du point $x_c$, donc l'éclairment reçu 
par le détecteur est proportionnelle à l'éclairement du point $x_c$. 

\subsection{}
Quand $L\gg\frac{1}{K}$, on a $LK\gg1$, donc $f(L)\sim0$, donc $P=\mathcal{E}_0Ll\left(1+f(L)\cos(Kx_c)\right)
\sim \boxed{\mathcal{E}_0Ll}$, $P$ est donc indépendante de $x_c$. On a aussi
$L\gg\lambda=\frac{2\pi}{K}$, la longueur caractéristique du détecteur est beaucoup plus grande que la longueur
d'onde. L'éclairement reçu est indépendant du chaque point sur le détecteur. Il peut bien détecter.


\end{CJK}
\end{document}