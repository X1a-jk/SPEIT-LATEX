%%% Ne pas modifier jusqu'à la ligne 25
\documentclass[a4paper,12pt]{book}
\usepackage[utf8]{inputenc}
\usepackage[french]{babel}
%%\usepackage{CJK}
\usepackage{yhmath}
\usepackage[left=2cm,right=2cm,top=3cm,bottom=2cm, headheight=1.5cm,headsep=1.5cm]{geometry}
%%\usepackage{CJKutf8}
\usepackage{amsfonts}
\usepackage{mathrsfs}
\usepackage{amsmath,amsfonts,amssymb,dsfont}
\usepackage{graphicx}
\usepackage{subfigure}
\usepackage{enumitem}		%\enumerate-resume
\usepackage[colorlinks=true,unicode={true},hyperindex=false, linkcolor=blue, urlcolor=blue]{hyperref}
\newcommand{\myref}[1]{\ref{#1} page \pageref{#1}}

\addto\captionsfrench{\def\tablename{Tableau}}  %légendes des tableaux
\renewcommand\thesection{\Roman{section}~-~} 
\renewcommand\thesubsection{\Roman{section}.\Alph{subsection}~-~} 
\renewcommand\thesubsubsection{\Roman{section}.\Alph{subsection}.\arabic{subsubsection}~-~} 

\newcommand{\conclusion}[1]{\newline \centerline{\fbox{#1}}}

\setcounter{secnumdepth}{3}
\parindent=0pt

\usepackage{fancyhdr}
\pagestyle{fancy}

\lhead{SJTU-ParisTech} 
%%%%%%%%%%%%%%%%%%%%%%%%%%%%%%%%%%
\chead{DM}
\rhead{Daniel 518261910024}
\begin{document}

\renewcommand{\labelitemi}{$\blacktriangleright$}
\renewcommand{\labelitemii}{$\bullet$}


\section{Exercice 1}
\subsection{}
Soit $x \in \mathring{\overline{U}} \cap \mathring{\overline{V}}=\mathring{\widehat{\overline{U} \cap \overline{V}}}$, alors il existe $ r>0$, tel que $BO(x,r) \subset \overline{U} \cap \overline{V}$, 

Soit $y \in BO(x,r) \cap U$, car elles sont ouverts de $(E,d)$, alors il existe$ r^{'} >0$ tel que $BO(y,r^{'}) \subset BO(x,r) \cap U$. 
Mais comme $U \cap V=\emptyset$, donc $y \in V \subset \overline{V}$, c'est absurde. 

Finalement, on a $x \in \emptyset$, donc $\boxed{\mathring{\overline{U}} \cap \mathring{\overline{V}}=\emptyset}$

\subsection{}
Par exemple, on se place dans $(\mathbb{R},\vert\cdot \vert)$, on pose $U=\mathbb{Q}$, $V=\mathbb{R}\backslash\mathbb{Q}$

On a $U \cap V=\emptyset$, mais $\mathring{\overline{U}}=\mathbb{R}$, $\mathring{\overline{V}}=\mathbb{R}$. Donc $\mathring{\overline{U}} \cap \mathring{\overline{V}}=\mathbb{R} \neq \emptyset$

\subsection{}
Soit $a \in \mathring{\partial U}$, alors il existe $r>0$ tel que $BO(a,r) \subset \partial U=\overline{U} \backslash U$ car $U$ est un ouvert de $(E,d)$. 

Donc $BO(a,r) \cap U=\emptyset$, donc $a \notin \overline{U}$, c'est l'absurde car $BO(a,r) \subset \overline{U}$. On en déduit que $\boxed{\mathring{\partial U}=\emptyset}$
\subsection{}

Soit $(u_n)_{n \in \mathbb{N}}$ une suite de $A+B$, alors pour tout $n \in \mathbb{N}$, il existe $a_n \in A$, $b_n \in B$ tels que 
$u_n=a_n+b_n$. On obtient donc une suite $(a_n)_{n \in \mathbb{N}}$ de A, et $(b_n)_{n \in \mathbb{N}}$ une suite de B. 

% Comme A et B sont compacts de $(E,\Vert\cdot\Vert)$, donc il existe deux extractions $\phi$ et $\psi$, et $\alpha \in A$, $\beta \in B$ tels que
% $$
% a_{\phi(n)} \xrightarrow[n \to +\infty]{}\alpha, b_{\psi(n)} \xrightarrow[n \to +\infty]{}\beta
% $$ 
On considère la fonction 
\begin{equation}  \nonumber
    f:\left\{  
                 \begin{array}{rcl}  
                  E  & \to & \mathbb{R}\\
                  x  & \mapsto& \Vert x\Vert
                 \end{array}  
    \right.  
\end{equation}
Car A et B sont compacts, donc ils sont bornés. Donc les suites $(f(a_n))_{n \in \mathbb{N}}$ et $(f(b_n))_{n \in \mathbb{N}}$ sont bornés.
 Soit $n \in \mathbb{N}$, on a 
$$
f(u_n)=\Vert u_n\Vert \leq \Vert a_n\Vert+\Vert b_n\Vert
$$
Donc $(f(u_n))_{n \in \mathbb{N}}$ est borné. 

D'après le théorème de Weierstrass, il existe $l \in \mathbb{R}$ et une extraction $\varphi$ tels que 
$$
f(u_{\phi(n)}) \xrightarrow[n \to +\infty]{}l
$$
On sait que $f$ est continue sur E, on a donc 
$$
\forall \epsilon>0, \exists N \in \mathbb{N}^*, \forall n >N \Rightarrow f(u_{\phi(n)}) \in BO(l,\epsilon)
$$
Donc 
$$
\forall \epsilon>0, \exists N \in \mathbb{N}^*, \forall n >N \Rightarrow u_{\phi(n)} \in f^{-1}(BO(l,\epsilon))
$$
Finalement, on a $u_{\phi(n)} \xrightarrow[n \to +\infty]{} f^{-1}(l)$, on a donc \fbox{A+B est compact}

\section{Exercice 2}
\subsection{}
Soit $((x_n,y_n))_{n \in \mathbb{N}}$ une suite de $(E^2, \delta)$, on a 
\begin{align*}
    & \quad ((x_n,y_n))_{n \in \mathbb{N}} \, \mbox{converge vers} (x,y) \\
    & \Leftrightarrow \delta((x_n,y_n),(x,y)) \xrightarrow[n \to +\infty]{}0\\
    & \Leftrightarrow max(d(x_n,x),d(y_n,y))\xrightarrow[n \to +\infty]{}0\\
    & \Leftrightarrow  d(x_n,x)\xrightarrow[n \to +\infty]{}0, d(y_n,y)\xrightarrow[n \to +\infty]{}0\\
    & \Leftrightarrow \boxed{x_n \xrightarrow[n \to +\infty]{} x, y_n \xrightarrow[n \to +\infty]{} y}
\end{align*}
\subsection{}
Si on prend $u_1 \in K_1$, alors il existe $v_1 \in K_2$ tel que $d(u_1,v_1)=d(u_1,K_2)$ car $K_2$ est un compact de  $(E,d)$. 

De même, il existe $u_2\in K_1$ tel que $d(v_1,u_2)=d(v_1,K_1) \leq d(v_1,u_1)$. Par récurrence, on peut construire $(u_n)_{n \in \mathbb{N}}$ une suite de $K_1$, 
$(v_n)_{n \in \mathbb{N}}$ une suite de $K_2$ tels que $(d(u_n,v_n))_{n \in \mathbb{N}}$ est décroissante, avec la borne inférieure $d(K_1,K_2)$, donc on a 
$$
d(u_n,v_n)\xrightarrow[n \to +\infty]{}d(K_1,K_2)
$$ 
Comme $K_1$ et $K_2$ sont des compacts de $(E,d)$, alors il existe deux extractions $\phi$ et $\psi$, et $x_1 \in K_1$, $x_2 \in K_2$ tels que
 $$
 u_{\phi(n)} \xrightarrow[n \to +\infty]{}x_1, v_{\phi \circ \psi(n)} \xrightarrow[n \to +\infty]{}x_2
 $$
On a aussi 
$$
d(u_{\phi \circ \psi(n)},v_{\phi \circ \psi(n)} ) \xrightarrow[n \to +\infty]{} d(x_1,x_2)
$$
Par l'unicité des limites, on a donc $\boxed{d(x_1,x_2)=d(K_1,K_2)}$
\subsection{}
\begin{itemize}
    \item Soit $x \in K$, alors $d(x,F)\geq d(K,F)$. On peut passer a la limite, donc $inf(\{d(x,F),x \in K\}) \geq d(K,F)$
    % \item il existe une suite $(u_n)_{n \in \mathbb{N}}$ de F tel que soit $x \in K$, $d(x,u_n)\xrightarrow[n \to +\infty]{}d(x,F)$. Donc on a 
    % $$
    % \{d(x,F),x \in K\} \subset \{d(x,f),x\in K,f \in F\}
    % $$
    % donc $inf(\{d(x,F),x \in K\})\leq inf(\{d(x,f),x \in K,f \in F\})=d(K,F)$
    \item Soient $x \in K$, $f \in F$, alors $d(x,F) \leq d(x,f)$. Ceci est vraie pour tout $x \in K$, on peut passer a la limite, donc 
    $$
    inf(\{d(x,F),x \in K\})\leq inf(\{d(x,f),x \in K,f \in F\})=d(K,F)
    $$
\end{itemize}
On obtient donc $inf(\{d(x,F),x \in K\})=d(K,F)$

On considère la fonction 
\begin{equation}  \nonumber
    f:\left\{  
                 \begin{array}{rcl}  
                  K  & \to & \mathbb{R}\\
                  x  & \mapsto& d(x,F)
                 \end{array}  
    \right.  
\end{equation}
On sait que $f$ est continue sur $K$. D'après le cours, comme $K$ est un compact de $(E,d)$, donc $f$ atteint ses bornes. 

Donc il existe $x_m \in K$ tel que 
$$
f(x_m,F)=inf(\{d(x,F),x \in K\})
$$ 
Supposons que $d(K,F)=0$, Donc il existe $x_m \in K$, tel que 
$d(x_m,F)=inf(\{d(x,F),x \in K\})=0$. Car $F$ est un fermé de $(E,d)$, donc $x_m \in F$. Mais comme $F \cap K=\emptyset$ d'après l'hypothèse, c'est absurde. 

On a donc $\boxed{d(K,F) \neq 0}$

\subsection{}
On se place dans $E=\mathbb{R} \backslash \{0\}$ muni de la distance $\vert\cdot\vert$, 
avec $K=[-1,0[$, $F=]0,1]$ deux fermés de $(E,\vert\cdot\vert)$. On a $F \cap K=\emptyset$, mais $d(K,F)=0$. 

(On peut prendre $(2^{-n})_{n \in \mathbb{N}}$ une suite de F, $(-2^{-n})_{n \in \mathbb{N}}$ une suite de K, $inf(\{2^{-n}-(-2^{-n}),n \in \mathbb{N}\})=0$)


 \end{document}