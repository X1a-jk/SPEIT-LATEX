%%% Ne pas modifier jusqu'à la ligne 25
\documentclass[a4paper,12pt]{book}
\usepackage[utf8]{inputenc}
\usepackage[french]{babel}
%%\usepackage{CJK}
\usepackage{yhmath}
\usepackage[left=2cm,right=2cm,top=3cm,bottom=2cm, headheight=1.5cm,headsep=1.5cm]{geometry}
%%\usepackage{CJKutf8}
\usepackage{amsfonts}
\usepackage{mathrsfs}
\usepackage{amsmath,amsfonts,amssymb,dsfont}
\usepackage{graphicx}
\usepackage{subfigure}
\usepackage{enumitem}		%\enumerate-resume
\usepackage[colorlinks=true,unicode={true},hyperindex=false, linkcolor=blue, urlcolor=blue]{hyperref}
\newcommand{\myref}[1]{\ref{#1} page \pageref{#1}}

\addto\captionsfrench{\def\tablename{Tableau}}  %légendes des tableaux
\renewcommand\thesection{\Roman{section}~-~} 
\renewcommand\thesubsection{\Roman{section}.\Alph{subsection}~-~} 
\renewcommand\thesubsubsection{\Roman{section}.\Alph{subsection}.\arabic{subsubsection}~-~} 

\newcommand{\conclusion}[1]{\newline \centerline{\fbox{#1}}}

\setcounter{secnumdepth}{3}
\parindent=0pt

\usepackage{fancyhdr}
\pagestyle{fancy}

\lhead{SJTU-ParisTech} 
%%%%%%%%%%%%%%%%%%%%%%%%%%%%%%%%%%
\chead{DM}
\rhead{Daniel 518261910024}
\begin{document}

\renewcommand{\labelitemi}{$\blacktriangleright$}
\renewcommand{\labelitemii}{$\bullet$}


\section{Exercice 1}
On note $U=\{u_n,n\in \mathbb{N}\} \cap \{l\}$, soit $(a_n)_{n \in \mathbb{N}}$ une suite de $U$. On va distinguer 2 cas.
\begin{itemize}
    \item Sit $\forall \epsilon>0, \exists N \in \mathbb{N}, n \geq N \Rightarrow d(a_n,l)\leq \epsilon$. On peut prendre $(\epsilon_n)_{n \in \mathbb{N}}$ tel que 
    $\forall n \in \mathbb{N}$, $\epsilon_n=2^{-n}$, on peut donc construire $\varphi(n)$ par récurrence:
    \begin{itemize}
        \item $n=0$, $\epsilon_0=1$. Par notre hypothèse, il existe $N_0 \in \mathbb{N}$ tel que $\forall n\geq N_0$, $d(a_n,l)\leq 1$. 
        On peut prendre $\varphi(0)=N_0$, on a bien $d(a_{\varphi(0),l})\leq \epsilon_0$
        \item Supposons que on a construit $\varphi(n)$. Pour le rang $n+1$, on a $\exists N_{n+1} \in \mathbb{N}$ tel que $\forall n \geq N_{n+1}$, $d(a_{n},l)\leq \epsilon_{n+1}=2^{-(n+1)}$. 
        On peut prendre $\varphi(n+1)=max(\varphi(n)+1,N_{n+1})$. Alors $\varphi(n+1)>\varphi(n)$, et $d(a_{\varphi(n+1)},l)\leq \epsilon_{n+1}$
    \end{itemize}
    On obtient donc une extraction $\varphi$ telle que $\forall n \in \mathbb{N}$, $d(a_{\varphi(n)},l)\leq \epsilon_n=2^{-n}$, donc
    $$
    d(a_{\varphi(n)},l)\xrightarrow[n \to +\infty]{}0 
    $$
    donc 
    $$
    a_\varphi(n)\xrightarrow[n \to +\infty]{}l   
    $$
    $(a_n)_{n \in \mathbb{N}}$ admet donc une valeur d'adhérence
    \item Sinon, c'est-à-dire $\exists \epsilon>0, \forall N \in \mathbb{N}, \exists n \geq N, d(a_n,l)> \epsilon$. On fixe tel $\epsilon$. 
    
    Comme on a 
    $$
    u_\varphi(n)\xrightarrow[n \to +\infty]{}l   
    $$
    alors pour cette $\epsilon$, il existe $N \in \mathbb{N}$ tel que $\forall n > N$, $d(u_n,l)< \epsilon$. Donc on a 
    $$
    card(\{n \in \mathbb{N}|d(u_n,l)>\epsilon\})\leq N+1
    $$
    le cardinal est donc fini. On fixe tel $N$

    Alors soit $n \geq N$ tel que $d(a_n,l) \geq \epsilon$  il existe $i \in [\![0,N]\!]$ tel que 
    $$
    card(\{n \in \mathbb{N}|a_n=u_i\})=+\infty
    $$
    car $(a_n)_{n \in \mathbb{N}}$ est une suite de $U$

    On peut donc construire une extraction $\varphi$ telle que 
    $$
    \forall n \in \mathbb{N}, \varphi(n)=u_i
    $$ 
    On a donc la suite-extraite  $(a_{\varphi(n)})_{n \in \mathbb{N}}$ converge vers $u_i$, $(a_n)_{n \in \mathbb{N}}$ admet donc une valeur d'adhérence. 
\end{itemize}

En tous cas, $(a_n)_{n \in \mathbb{N}}$ admet donc une valeur d'adhérence. Donc $U$ est un compact de $(E,d)$
\section{Exercice 2}
\subsection{}
Soit $(y_n)_{n \in \mathbb{N}}$ une suite de $f(F)$ qui converge vers $y \in E$, montrons que $y \in f(F)$. 

Soit $n \in \mathbb{N}$, or $y_n \in f(F)$, il existe $x_n \in F$ tel que $y_n=f(x_n)$. Selon l'exercice $1$, on a 
$Y=\{y_n,n \in \mathbb{N}\}\cup \{y\}$ est un compact de $(E,d)$. Comme $f$ est propre, on a $X=f^{-1}(Y)$ est un compact de $(E,d)$. Comme 
$(x_n)_{n \in \mathbb{N}}$ est une suite de $X$, alors il existe une extraction $\varphi$ est $x \in E$ tels que 
$$
x_{\varphi(n)}\xrightarrow[n \to +\infty]{}x
$$
De plus, $x \in F$ car $F$ est un fermé de $(E,d)$. Comme $f$ est continue, on a donc 
$$
f(x_{\varphi(n)})=y_{\varphi(n)}\xrightarrow[n \to +\infty]{}f(x) \in f(F)
$$ 
$f(x)$ est donc une valeur d'adhérence de $(y_n)_{n \in N}$. Comme $(y_n)_{n \in N}$ converge, la valeur d'adhérence est unique, donc $y=f(x)\in f(F) $. 

On en déduit que \fbox{$f(F)$ est un fermé de $(E,d)$}

\subsection{}
\begin{itemize}
    \item sens direct: Soient $f$ est propre et $M\geq 0$, alors comme $BF(0,M)$ est un compact de $(E,\Vert\cdot\Vert)$, donc 
    $f^{-1}(BF(0,M))$ est un conpact par notre hypothèse, il est donc borné. Donc il existe $m\geq 0$ tel que $f^{-1}(BF(0,M)) \subset BF(0,m)$. 
    On en déduit que 
    $$
    f(x) \in BF(0,M) \Rightarrow x \in BF(0,m)
    $$ 
    c'est-à-dire 
    $$
    x \notin BF(0,m) \Rightarrow f(x) \notin BF(0,M)
    $$
    On a donc 
    $$\forall M \geq 0, \exists m \geq 0, \Vert x\Vert > m \Rightarrow \Vert f(x)\Vert > M  $$

    C'est équivalent à 
    $$
    \Vert f(x) \Vert \xrightarrow[\Vert x \Vert \to +\infty]{} +\infty
    $$
    \item sens indirect: Soit $K$ est un compact de $(E,\Vert\cdot\Vert)$, il est donc fermé et borné de $(E,\Vert\cdot\Vert)$. Comme $f$ est continue, $f^{-1}(K)$ est donc un fermé de $(E,\Vert\cdot\Vert)$
    
    Montrons que $f^{-1}(K)$ est borné de $(E,\Vert\cdot\Vert)$. 

    Or $K$ est borné, il existe $M>0$ tel que $K \subset BO(0,M)$. Car $\Vert f(x) \Vert \xrightarrow[\Vert x \Vert \to +\infty]{} +\infty$, donc il existe 
    $m>0$ tel que $f(x) \subset BO(0,m)\Rightarrow x \subset BO(0,m)$, ce qui implique que $f^{-1}(BO(0,M))\subset BO(0,m)$. 

    On a donc $f^{-1}(K) \subset f^{-1}(BO(0,M))\subset BO(0,m)$, il est donc borné. On en déduit que $f^{-1}(K)$ est un compact de $(E,\Vert\cdot\Vert)$
\end{itemize}
Finalement, on en déduit que 
$$
f \,\mbox{est propre} \Leftrightarrow \Vert f(x) \Vert \xrightarrow[\Vert x \Vert \to +\infty]{} +\infty
$$
\section{Exercice 3}
\subsection{}
Soient $(f,g) \in E^2$, $\lambda \in \mathbb{R}$, donc soit $x \in [0,1]$, on a 
\begin{align*}
    \phi(f+\lambda g)(x)&=\int_0^x t (f+\lambda g)(t)\,dt+x\int_x^1(f+\lambda g)(t)\,dt\\
    &=\int_0^x t f(t)\,dt+x\int_x^1f(t)\,dt+\lambda\left(\int_0^x t g(t)\,dt+x\int_x^1g(t)\,dt\right)\\
    &=\phi(f)(x)+\lambda\phi(g)(x)
\end{align*}
On a donc $\phi(f+\lambda g)=\phi(f)+\lambda\phi(g)$

De plus $x \mapsto \int_0^x t f(t)\,dt$ et $x \mapsto \int_x^1 f(t)\,dt$ sont continues car elles sont dérivables. 
Donc $\phi(f) \in E$  car tous ses composants sont continues par addition et multiplication. Donc $\boxed{\phi \in \mathcal{L}(E)}$

\subsection{}
Soit $f \in E$, comme $f$ est continue sur $[0,1]$, sa borne supérieure est atteinte, on la note $A=\sup_{x \in [0,1]}|f(x)|$.

Soit $x \in [0,1]$ fixé, soit $t \in [0,x]$, on a donc $|tf(t)|\leq tA$. En intégrant, on a 
$$
\left|\int_0^x tf(t)\,dt\right| \leq\int_0^x |tf(t)|\,dt \leq \int_0^x tA\,dt=A\int_0^x t\,dt=\frac{1}{2}Ax^2
$$
De même, 
$$
\left|x\int_x^1f(t)\,dt\right|\leq x\int_x^1A\,dt=Ax(1-x)
$$
On a donc 
$$
\left|\int_0^x t f(t)\,dt+x\int_x^1f(t)\,dt\right| \leq \left|\int_0^x t f(t)\,dt\right|+\left|x\int_x^1f(t)\,dt\right| \leq (x-\frac{1}{2}x^2)A
$$
cette inégalité est valide pour tout $x \in [0,1]$, on a donc $x-\frac{1}{2}x^2\leq \frac{1}{2}$, donc 
$$
\forall x \in [0,1],\left|\int_0^x t f(t)\,dt+x\int_x^1f(t)\,dt\right| \leq \frac{1}{2}A
$$
En passant à la limite, on obtient 
$$
\sup_{x \in [0,1]}\left(\left|\int_0^x t f(t)\,dt+x\int_x^1f(t)\,dt\right|\right) \leq \frac{1}{2}A
$$
Donc on a 
$$
\Vert \phi(f)\Vert_\infty \leq \frac{1}{2} \Vert f \Vert_\infty
$$
Donc $\phi$ est continue sur E car elle est une application linéaire. 

\subsection{}
Comme $\phi$ est continue sur E
$$
|||\phi|||=\sup_{f \in E \backslash \{0_E\}} \frac{\Vert \phi(f)\Vert_\infty}{\Vert f \Vert_\infty}
$$
est bien définie, et on a $|||\phi|||\leq \frac{1}{2}$. 

Si on prend $f: t\in[0,1] \mapsto 1$, on a $f\in E$, et $\Vert f \Vert_\infty=\sup_{t \in [0,1]}f(t)=1$. 

De plus, soit $x \in [0,1]$
$$
\phi(f(x))=\int_0^x t\,dt+x\int_x^1\,dt=x-\frac{1}{2}x^2
$$
Donc $\Vert \phi(f)\Vert_\infty=\sup_{x \in [0,1]}\phi(f(x))=\frac{1}{2}$. On a donc $\frac{\Vert \phi(f)\Vert_\infty}{\Vert f \Vert_\infty}=\frac{1}{2}$, donc $|||\phi|||\geq \frac{1}{2}$. 

Par double inégalité, on obtient $\boxed{|||\phi|||= \frac{1}{2}}$


 \end{document}