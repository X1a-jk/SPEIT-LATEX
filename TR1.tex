%%% Ne pas modifier jusqu'à la ligne 25
\documentclass[a4paper,12pt]{book}
\usepackage[utf8]{inputenc}
\usepackage[french]{babel}
%%\usepackage{CJK}
\usepackage{yhmath}
\usepackage[left=2cm,right=2cm,top=3cm,bottom=2cm, headheight=1.5cm,headsep=1.5cm]{geometry}
%%\usepackage{CJKutf8}
\usepackage{amsfonts}
\usepackage{amsmath,amsfonts,amssymb,dsfont}
\usepackage{graphicx}
\usepackage{enumitem}		%\enumerate-resume
\usepackage[colorlinks=true,unicode={true},hyperindex=false, linkcolor=blue, urlcolor=blue]{hyperref}
\newcommand{\myref}[1]{\ref{#1} page \pageref{#1}}

\addto\captionsfrench{\def\tablename{Tableau}}  %légendes des tableaux
\renewcommand\thesection{\Roman{section}~-~} 
\renewcommand\thesubsection{\Roman{section}.\Alph{subsection}~-~} 
\renewcommand\thesubsubsection{\Roman{section}.\Alph{subsection}.\arabic{subsubsection}~-~} 

\newcommand{\conclusion}[1]{\newline \centerline{\fbox{#1}}}

\setcounter{secnumdepth}{3}
\parindent=0pt

\usepackage{fancyhdr}
\pagestyle{fancy}

\lhead{SJTU-ParisTech} 
%%%%%%%%%%%%%%%%%%%%%%%%%%%%%%%%%%
\chead{TR1}
\rhead{Daniel 518261910024}

\begin{document}
\renewcommand{\labelitemi}{$\blacktriangleright$}
\renewcommand{\labelitemii}{$\bullet$}

\section{les conditions pour obtenir des interférences lumineuses}
Au point M étudié, soit $s_1(M,t)=a_1(M)\cos(\omega t-\phi_1(M))$, $s_2(M,t)=a_2(M)\cos(\omega t-\phi_2(M))$ les deux signaux.
On sait que le terme d'interférences est 
\begin{equation}
2\sqrt{\mathcal{E}_1(M)\mathcal{E}_2(M)}(<\cos((\omega_1+\omega_2)t-(\phi_1+\phi_2))>+<\cos((\omega_1-\omega_2)t-(\phi_1-\phi_2))>) \label{e1}
\end{equation}
\subsection{pulsation temporelle}
Il faux que les deux signaux soient de même pulsation: $\boxed{\omega_1=\omega_2}$, sinon, equation(1) vaut $0$: pas de interférences.  
(car la valeur moyenne d'une équation cosinusoïdal de pulsation non nulle dans une période vaut $0$). On le note $S$.
Ainsi, équation(1) devient 
\begin{equation}
    2\sqrt{\mathcal{E}_1(M)\mathcal{E}_2(M)}<cos(\Delta\phi_{2/1}(M))>
\end{equation}
où $\Delta\phi_{2/1}(M)=\phi_2(M)-\phi_1(M)$, le déphasage entre les deux signaux. 
\subsection{source}
Il faut que les deux signaux \fbox{issus de même source}. Sinon, selon le modèle des trains d'ondes, équation(2) vaut 0 (car 
$\Delta\phi_{2/1}(M)$ change aléatoirement, sa valeur moyenne vaut 0)
\subsection{longueur de cohérence}
selon le modèle des trains d'ondes, les deux signaux doit être associés à le même train d'onde. Il faut que 
$\frac{|(SM)_2-(SM)_1|}{c}<\tau_c$, avec $\tau_c$ le temps de cohérence. On a donc $\boxed{|\delta_{2/1}(M)|<l_c}$, avec
 $l_c=c*\tau_c$ la longueur de cohérence, $\delta_{2/1}(M)=(SM)_2-(SM)_1$ la différence de marche. 
\subsection{éclairement comparable}
En notant $C=2\frac{\sqrt{r}}{r+1}$, où $r=\frac{\mathcal{E}_1(M)}{\mathcal{E}_2(M)}$, l'éclairement totale 
\begin{align*}
    \mathcal{E}(M)&=\mathcal{E}_1(M)+\mathcal{E}_2(M)+2\sqrt{\mathcal{E}_1(M)\mathcal{E}_2(M)}<cos(\Delta\phi_{2/1}(M))>\\
                  &=\mathcal{E}_1(M)+\mathcal{E}_2(M))(1+C<cos(\Delta\phi_{2/1}(M))>
\end{align*}
lorsque $\mathcal{E}_1(M) \gg \mathcal{E}_2(M)$ ou $ \mathcal{E}_1(M)\ll \mathcal{E}_2(M)$, $C\rightarrow 0$, 
on a $\mathcal{E}(M)\sim \mathcal{E}_1(M)+\mathcal{E}_2(M)$, les interférences sont invisibles. Donc les éclairements \fbox{doivent être comparables}. 
\end{document}