%%% Ne pas modifier jusqu'à la ligne 25
\documentclass[a4paper,12pt]{book}
\usepackage[utf8]{inputenc}
\usepackage[french]{babel}
%%\usepackage{CJK}
\usepackage{yhmath}
\usepackage[left=2cm,right=2cm,top=3cm,bottom=2cm, headheight=1.5cm,headsep=1.5cm]{geometry}
%%\usepackage{CJKutf8}
\usepackage{amsfonts}
\usepackage{mathrsfs}
\usepackage{amsmath,amsfonts,amssymb,dsfont}
\usepackage{graphicx}
\usepackage{subfigure}
\usepackage{enumitem}		%\enumerate-resume
\usepackage[colorlinks=true,unicode={true},hyperindex=false, linkcolor=blue, urlcolor=blue]{hyperref}
\newcommand{\myref}[1]{\ref{#1} page \pageref{#1}}

\addto\captionsfrench{\def\tablename{Tableau}}  %légendes des tableaux
\renewcommand\thesection{\Roman{section}~-~} 
\renewcommand\thesubsection{\Roman{section}.\Alph{subsection}~-~} 
\renewcommand\thesubsubsection{\Roman{section}.\Alph{subsection}.\arabic{subsubsection}~-~} 

\newcommand{\conclusion}[1]{\newline \centerline{\fbox{#1}}}

\setcounter{secnumdepth}{3}
\parindent=0pt

\usepackage{fancyhdr}
\pagestyle{fancy}

\lhead{SJTU-ParisTech} 
%%%%%%%%%%%%%%%%%%%%%%%%%%%%%%%%%%
\chead{DM1}
\rhead{Daniel 518261910024}

\begin{document}
\renewcommand{\labelitemi}{$\blacktriangleright$}
\renewcommand{\labelitemii}{$\bullet$}


\section{Activité 1-3}
Dans le référentiel $(R)$, le moment cinétique du système $(\Sigma)$ par rapport à $A$ est défini par 
$$
\overrightarrow{L}_{A,(R)}(\Sigma)=\int_{(\Sigma)}\overrightarrow{AP} \wedge \vec{v}_{(R)}(P)\,dm_P
$$
Si on changement de point de référence $B$ avec $\overrightarrow{AP}=\overrightarrow{AB}+\overrightarrow{BP}$, on obtient 
\begin{align*}
\overrightarrow{L}_{A,(R)}(\Sigma)&=\int_{(\Sigma)}(\overrightarrow{AB}+\overrightarrow{BP}) \wedge \vec{v}_{(R)}(P)\,dm_P\\
&=\int_{(\Sigma)}\overrightarrow{BP} \wedge \vec{v}_{(R)}(P)\,dm_P+\int_{(\Sigma)}\overrightarrow{AB} \wedge \vec{v}_{(R)}(P)\,dm_P\\
&=\overrightarrow{L}_{B,(R)}(\Sigma)+\overrightarrow{AB}\wedge\int_{(\Sigma)}\vec{v}_{(R)}(P)\,dm_P
\end{align*}
Par la définition de quantité de mouvement $\vec{p}_{(R)}(\Sigma)=\int_{(\Sigma)}\vec{v}_{(R)}(P)\,dm_P$, 
on obtient 
$$\boxed{\overrightarrow{L}_{A,(R)}(\Sigma)=\overrightarrow{L}_{B,(R)}(\Sigma)+\overrightarrow{AB}\wedge\vec{p}_{(R)}(\Sigma)} $$

\section{Activité 1-4}
Dans le référentiel $(R)$, si on change deux points $A$ et $A^{'}$ sur l'axe $\Delta(A,\vec{u})$, d'après le résultat précédent, on a 
$$
\overrightarrow{L}_{A,(R)}(\Sigma)\cdot \vec{u}=\overrightarrow{L}_{A^{'},(R)}(\Sigma)\cdot \vec{u}+\overrightarrow{AA^{'}}\wedge \vec{p}_{(R)}(\Sigma)\cdot\vec{u}
$$
Par les calculs vectoriels $\vec{a}\cdot(\vec{b}\wedge\vec{c})=\vec{c}\cdot(\vec{a}\wedge\vec{b})$, on a alors 
$$
\overrightarrow{AA^{'}}\wedge \vec{p}_{(R)}(\Sigma)\cdot\vec{u}=(\vec{u}\wedge\overrightarrow{AA^{'}})\cdot\vec{p}_{(R)}(\Sigma)=0
$$
car $\overrightarrow{AA^{'}}$ est parallèle à $\vec{u}$. 

On a donc $\boxed{\overrightarrow{L}_{A,(R)}(\Sigma)\cdot \vec{u}=\overrightarrow{L}_{A^{'},(R)}(\Sigma)\cdot \vec{u}}$. 

On peut donc définir le moment cinétique de $(\Sigma)$ relativement à l'axe $\Delta$ par 
$$
\overrightarrow{L}_{\Delta,(R)}(\Sigma)=\overrightarrow{L}_{A,(R)}(\Sigma)\cdot \vec{u}
$$
qui est indépendant du point choisi sur l'axe


\end{document}