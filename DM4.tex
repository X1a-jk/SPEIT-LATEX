%%% Ne pas modifier jusqu'à la ligne 25
\documentclass[a4paper,12pt]{book}
\usepackage[utf8]{inputenc}
\usepackage[french]{babel}
%%\usepackage{CJK}
\usepackage{yhmath}
\usepackage[left=2cm,right=2cm,top=3cm,bottom=2cm, headheight=1.5cm,headsep=1.5cm]{geometry}
%%\usepackage{CJKutf8}
\usepackage{amsfonts}
\usepackage{mathrsfs}
\usepackage{amsmath,amsfonts,amssymb,dsfont}
\usepackage{graphicx}
\usepackage{subfigure}
\usepackage{enumitem}		%\enumerate-resume
\usepackage[colorlinks=true,unicode={true},hyperindex=false, linkcolor=blue, urlcolor=blue]{hyperref}
\newcommand{\myref}[1]{\ref{#1} page \pageref{#1}}

\addto\captionsfrench{\def\tablename{Tableau}}  %légendes des tableaux
\renewcommand\thesection{\Roman{section}~-~} 
\renewcommand\thesubsection{\Roman{section}.\Alph{subsection}~-~} 
\renewcommand\thesubsubsection{\Roman{section}.\Alph{subsection}.\arabic{subsubsection}~-~} 

\newcommand{\conclusion}[1]{\newline \centerline{\fbox{#1}}}

\setcounter{secnumdepth}{3}
\parindent=0pt

\usepackage{fancyhdr}
\pagestyle{fancy}

\lhead{SJTU-ParisTech} 
%%%%%%%%%%%%%%%%%%%%%%%%%%%%%%%%%%
\chead{DM4}
\rhead{Daniel 518261910024}

\begin{document}
\renewcommand{\labelitemi}{$\blacktriangleright$}
\renewcommand{\labelitemii}{$\bullet$}


\section{Enthalpie libre standard de la synthèse de l’eau}
\subsection{}
à $T_1=298K$, on a $\Delta_rS^\circ(T_1)=\sum_i\nu_iS^\circ_{m,i}(T_1)=S^\circ_{m,H_2O}(T_1)-\frac{1}{2}S^\circ_{m,O_2}(T_1)-S^\circ_{m,H_2}(T_1)$

\hspace*{\fill} 

A.N. $\Delta_rS^\circ(T_1)=188.7-\frac{1}{2}*205.0-130.6=-44.40 J\cdot K^{-1}\cdot mol^{-1}$

\hspace*{\fill} 

Alors on a $\boxed{\Delta_rG^\circ(T_1)=\Delta_rH^\circ(T_1)-T_1\Delta_rS^\circ(T_1)}$

\hspace*{\fill} 

A.N. $\boxed{\Delta_rG^\circ(T_1)=-241.8*10^3-298*(-44.40)=-2.29*10^5\,J\cdot mol^{-1}}$
    
                       
\subsection{}

Les capacités thermiques molaires standard sont supposées indépendantes de T, 
on a donc $\Delta_rC^\circ_{p}=C^\circ_{p,m,H_2O}-\frac{1}{2}C^\circ_{p,m,O_2}-C^\circ_{p,m,H_2}$

\hspace*{\fill} 

A.N. $\Delta_rC^\circ_{p}=33.6-\frac{1}{2}*29.4-28.8=-9.90\, J\cdot K^{-1} \cdot mol^{-1}$
\subsubsection{}
à $T_2=1000K$, par les apprximations d'Ellinghams, on a 
$$
\Delta_rH^\circ(T_2)=\Delta_rH^\circ(T_1)+\int^{T_2}_{T_1}\Delta_rC^\circ_p\,dT
$$
$$
\Delta_rS^\circ(T_2)=\Delta_rS^\circ(T_1)+\int^{T_2}_{T_1}\frac{\Delta_rC^\circ_p}{T}\,dT
$$
Les capacités thermiques molaires standard sont supposées indépendantes de T, on a donc 
$$
\Delta_rH^\circ(T_2)=\Delta_rH^\circ(T_1)+({T_2}-{T_1})\Delta_rC^\circ_p
$$
$$
\Delta_rS^\circ(T_2)=\Delta_rS^\circ(T_1)+\ln\frac{T_2}{T_1}\Delta_rC^\circ_p
$$
A.N. 
$$\Delta_rH^\circ(T_2)=-241.8*10^3+(1000-298)*(-9.90)=-2.49*10^5\,J\cdot mol^{-1}$$
$$\Delta_rS^\circ(T_2)=-44.40+\ln\frac{1000}{298}*(-9.90)=-56.4\,J\cdot K^{-1}\cdot mol^{-1}$$
On a donc $\boxed{\Delta_rG^\circ(T_2)=\Delta_rH^\circ(T_2)-T_2\Delta_rS^\circ(T_2)}$

\hspace*{\fill} 

A.N. $\boxed{\Delta_rG^\circ(T_2)=-2.49*10^5-1000*(-56.4)=-1.93*10^5 \,J\cdot mol^{-1}}$
\subsubsection{}
On a aussi 
$$
\Delta_rG^\circ(T_2)=\Delta_rG^\circ(T_1)-\int^{T_2}_{T_1}\Delta_rS^\circ(T)\,dT
$$
où
$$
\Delta_rS^\circ(T)=\Delta_rS^\circ(T_1)-\ln\frac{T}{T_1}\Delta_rC^\circ_p
$$
donc
$$
\Delta_rG^\circ(T_2)=\Delta_rG^\circ(T_1)-({T_2}-{T_1})\Delta_rS^\circ(T_1)-\Delta_rC^\circ_p\left[T \ln T-T-T\ln T_1\right]^{T_2}_{T_1}
$$
A.N. $\boxed{\Delta_rG^\circ(T_2)=-1.93*10^5\,J\cdot mol^{-1}}$

\subsubsection{}
On a aussi
$$
\frac{\Delta_rG^\circ(T_2)}{T_2}=\frac{\Delta_rG^\circ(T_1)}{T_1}-\int^{T_2}_{T_1}\frac{\Delta_rH^\circ(T)}{T^2}\,dT
$$
avec
$$
\Delta_rH^\circ(T)=\Delta_rH^\circ(T_1)+(T-T_1)\Delta_rC_p^\circ
$$
On a donc 
$$
\Delta_rG^\circ(T_2)=T_2*\left(\frac{\Delta_rG^\circ(T_1)}{T_1}-\left[\frac{-\Delta_rH^\circ(T_1)}{T}\right]^{T_2}_{T_1}-\Delta_rC_p^\circ
\left[\ln T\right]^{T_2}_{T_1}+T_1\Delta_rC_p^\circ\left[\frac{-1}{T}\right]^{T_2}_{T_1}\right)
$$
A.N. $\boxed{\Delta_rG^\circ(T_2)=-1.93*10^5\,J\cdot mol^{-1}}$

\hspace*{\fill} 

Par les trois méthodes, on a toujours le meme resultat.
\end{document}