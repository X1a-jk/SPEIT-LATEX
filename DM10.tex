%%% Ne pas modifier jusqu'à la ligne 25
\documentclass[a4paper,12pt]{book}
\usepackage[utf8]{inputenc}
\usepackage[french]{babel}
%%\usepackage{CJK}
\usepackage{yhmath}
\usepackage[left=2cm,right=2cm,top=3cm,bottom=2cm, headheight=1.5cm,headsep=1.5cm]{geometry}
%%\usepackage{CJKutf8}
\usepackage{amsfonts}
\usepackage{amsmath,amsfonts,amssymb,dsfont}
\usepackage{graphicx}
\usepackage{enumitem}		%\enumerate-resume
\usepackage[colorlinks=true,unicode={true},hyperindex=false, linkcolor=blue, urlcolor=blue]{hyperref}
\newcommand{\myref}[1]{\ref{#1} page \pageref{#1}}

\addto\captionsfrench{\def\tablename{Tableau}}  %légendes des tableaux
\renewcommand\thesection{\Roman{section}~-~} 
\renewcommand\thesubsection{\Roman{section}.\Alph{subsection}~-~} 
\renewcommand\thesubsubsection{\Roman{section}.\Alph{subsection}.\arabic{subsubsection}~-~} 

\newcommand{\conclusion}[1]{\newline \centerline{\fbox{#1}}}

\setcounter{secnumdepth}{3}
\parindent=0pt

\usepackage{fancyhdr}
\pagestyle{fancy}

\lhead{SJTU-ParisTech} 
%%%%%%%%%%%%%%%%%%%%%%%%%%%%%%%%%%
\chead{DM10}
\rhead{Daniel 518261910024}

\begin{document}
\renewcommand{\labelitemi}{$\blacktriangleright$}
\renewcommand{\labelitemii}{$\bullet$}


\section{Exercice 1}
% \begin{table}[h]
% \begin{center}
%     \begin{tabular}{l|ccccccc}
%     \hline
%                       & $CH_4$      & + & $Br_2$       & = & $CH_3Br$ & + & $HBr$ \\ \hline
%         $c_{initial}(mol\,L^{-1})$ & $c_{1,i}=1.0*10^{-2}$       &   & $c_{2,i}=1.0*10^{-2}$      &   & $0$ &  & $0$\\ 
%         $c_{final}(mol\,L^{-1})$      & $c_{1,f}=1.0*10^{-2}-\xi_v$  &   & $c_{2,f}=1.0*10^{-2}-\xi_v$  &   & $\xi_v$ & & $\xi_v$\\ 
%     \end{tabular}
% \end{center}
% \end{table}
\subsection{}
On appliquer l’AEQS à $Br^\bullet$:
\begin{equation}
    \frac{d[Br^\bullet]}{dt}=0=2v_1-v_2+v_{-2}+v_3-2v_4
\end{equation}
On appliquer l’AEQS à $CH_3^\bullet$:
\begin{equation}
    \frac{d[CH_3^\bullet]}{dt}=0=v_2-v_{-2}-v_3
\end{equation}
\subsection{}
Selon les mécanismes, on a 
$$
\begin{array}{lcl}
    v_1(t) & = & k_1[Br_2]\\
    v_2(t) & = & k_2[Br^\bullet][CH_4]\\
    v_{-2}(t) & = & k_{-2}[CH_3^\bullet][HBr]\\
    v_3(t) & = & k_3[CH_3^\bullet][Br_2]\\
    v_4(t) & = & k_4[Br^\bullet]^2
\end{array}
$$
Selon $(1)+(2)$, on a $v_1=v_4$, donc 
$$
\boxed{[Br^\bullet]=\sqrt{\frac{k_1}{k_4}[Br_2]}}
$$
Selon $(2)$, on a $v_2-v_{-2}-v_3=0$, donc 
$$
k_2[Br^\bullet][CH_4]=k_{-2}[CH_3^\bullet][HBr]+k_3[CH_3^\bullet][Br_2]
$$
donc 
$$
k_2\sqrt{\frac{k_1}{k_4}[Br_2]}[CH_4]=[CH_3^\bullet](k_{-2}[HBr]+k_3[Br_2])
$$
d'où
$$
\boxed{[CH_3^\bullet]=\frac{k_2\sqrt{\frac{k_1}{k_4}[Br_2]}[CH_4]}{k_{-2}[HBr]+k_3[Br_2]}}
$$
\subsection{}
On a 
$$
v=\frac{d[CH_3Br]}{dt}=v_3=k_3[CH_3^\bullet][Br_2]=\frac{k_2\sqrt{\frac{k_1}{k_4}[Br_2][CH_4]}}{1+\frac{k_{-2}[HBr]}{k_3[Br_2]}}
$$
à l'initial, on a $[HBr]\simeq 0$, donc 
$$\boxed{v\simeq k_2\sqrt{\frac{k_1}{k_4}[Br_2]}[CH_4]}$$
\fbox{La réaction a un ordre initial}.
\subsection{}
On suppose la vitesse initiale de cette réaction est donnée par $v_{ini}=k[CH_4]^p[Br_2]^q$, avec $p,q$ les 
ordres initiaux partiels. Par les expériences, on obtient 
$$
\begin{array}{lcl}
    v_0 & = & kc_0^pc_0^q\\
    2v_0 & = & kc_0^p(4c_0)^q\\
    2v_0 & = & k(2c_0)^pc_0^q
\end{array}
$$
On obtient donc $q=\frac{1}{2}$, $p=1$, donc la loi de vitesse initiale de cette réaction:
$$
\boxed{v_{ini}=k[CH_4]\sqrt{[Br_2]}}
$$
\subsection{}

ces faits sont d'accord avec la loi trouvée dans la qustion $3$. On retrouve la loi 
de vitesse expérientale avec $k=k_2\sqrt{\frac{k_1}{k_4}}$

\hspace*{\fill} 

\end{document}