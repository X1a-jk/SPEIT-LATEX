%%% Ne pas modifier jusqu'à la ligne 25
\documentclass[a4paper,12pt]{book}
\usepackage[utf8]{inputenc}
\usepackage[french]{babel}
%%\usepackage{CJK}
\usepackage{yhmath}
\usepackage[left=2cm,right=2cm,top=3cm,bottom=2cm, headheight=1.5cm,headsep=1.5cm]{geometry}
%%\usepackage{CJKutf8}
\usepackage{amsfonts}
\usepackage{mathrsfs}
\usepackage{amsmath,amsfonts,amssymb,dsfont}
\usepackage{graphicx}
\usepackage{subfigure}
\usepackage{enumitem}		%\enumerate-resume
\usepackage[colorlinks=true,unicode={true},hyperindex=false, linkcolor=blue, urlcolor=blue]{hyperref}
\newcommand{\myref}[1]{\ref{#1} page \pageref{#1}}

\addto\captionsfrench{\def\tablename{Tableau}}  %légendes des tableaux
\renewcommand\thesection{\Roman{section}~-~} 
\renewcommand\thesubsection{\Roman{section}.\Alph{subsection}~-~} 
\renewcommand\thesubsubsection{\Roman{section}.\Alph{subsection}.\arabic{subsubsection}~-~} 

\newcommand{\conclusion}[1]{\newline \centerline{\fbox{#1}}}

\setcounter{secnumdepth}{3}
\parindent=0pt

\usepackage{fancyhdr}
\pagestyle{fancy}

\lhead{SJTU-ParisTech} 
%%%%%%%%%%%%%%%%%%%%%%%%%%%%%%%%%%
\chead{DM}
\rhead{Daniel 518261910024}

\begin{document}
\renewcommand{\labelitemi}{$\blacktriangleright$}
\renewcommand{\labelitemii}{$\bullet$}


\section{Exercice 0}
Soit $(G, ∗)$ un groupe fini de cardinal n, et soit $x \in G$, 
on a le sous-groupe $H=\langle x \rangle$ est un sous-groupe de G.

D'après le théorème de Lagrange, on a $card(H)|card(G)$. En notant $d=o(x)$ l'ordre de $x$, 
on a $d=card(H)$, donc $d|n$. D'après le deuxième caractérisation de l'ordre, on a donc $x^n=e$. 

Finalement, on a $\boxed{\forall x \in G, x^n=e}$
\section{Exercice 1}
\subsection{}
Soient $x \in G, y\in H$, on note $p=o(x)$, $q=o(y)$, $d=ppcm(p,q)$. On a donc $p|d$, $q|d$. Alors il existe 
$(m,n) \in \mathbb{Z}^2$ tel que $d=m*p=n*q$ 
\begin{itemize}
    \item On a $\forall n \in \mathbb{N}, (x,y)^n=(x^n,y^n)$(on peut le montrer par récurrence sur $n$). 
    
    On a donc $(x,y)^d=(x^d,y^d)=((x^p)^m,(y^q)^n)=(e_G^m,e_H^n)=(e_G,e_H)$ l'élément neutre du groupe produit.
    \item Soit $k \in \mathbb{Z}$ tel que $(x,y)^k=(e_G,e_H)$, on a donc $(x^k,y^k)=(e_G,e_H)$. 
    
    Alors $x^k=e_G$, $y^k=e_H$. D'après le deuxième caractérisation de l'ordre, on a donc $p|k$, $q|k$, donc 
    $d|k$
\end{itemize}
D'après le deuxième caractérisation de l'ordre, on a donc $\boxed{d=ppcm(o(x),o(y))=o(x,y)}$
\subsection{}
Soient $g \in G$ un générateur de $G$, $h \in H$ un générateur de $H$. On note $a=o(g)$, $b=o(h)$, les ordres respectivement de $g$ et de $h$.

\begin{itemize}
    \item sens indirect: Soit $a$ et $b$ sont premiers entre eux, soit $(m,n) \in G \times H$. 
    alors  il existe $M\in \mathbb{Z}$ tel que $m=g^M$, donc pour tout $x \in \mathbb{Z}$ tel que $ [x]_a=[M]_a$, on a toujours $g^x=m$ car $g^a=e_G$. 

    De même, il existe $N\in \mathbb{Z}$ tel que $n=h^N$, donc pour tout $y \in \mathbb{Z}$ tel que $[y]_b=[N]_b$, on a toujours $h^y=n$ car $h^b=e_H$. 

    Puisque $a$ et $b$ sont premiers entre eux, par le théorème chinois, il existe $X\in \mathbb{Z}$ tel que $\psi([X]_{ab})=([X]_{a},[X]_{b})=([M]_{a},[N]_{b})$
    (car $\psi$ est un isomorphisme). 
    
    On a donc $(g,h)^X=(g^X,h^X)=(g^M,h^N)=(m,n)$. $G \times H$ est donc monogène. Car $G$ et $H$ sont finis, $G\times H$ aussi, et il est donc \fbox{cyclique}
    \item sens direct: On va le montrer par l'absurde
    
    Soit $d=pgcd(a,b)>1$ et supposons que $G \times H$ est cyclique. 
    Alors $\exists(a_1,b_1) \in \mathbb{Z}^2$, $a=a_1d,b=b_1d$, avec $a_1,b_1$ premiers entre eux. On suppose que $(g,h) \in G \times H$ est un générateur. 

    Car $(e_G,h) \in G \times H$, alors il existe $X \in \mathbb{Z}$, tel que $(g,h)^X=(e_G,h)$. Donc $e_G=g^X, h=h^X$. 
    Car $G$ est $\mathbb{Z}/a\mathbb{Z}$ sont isomorphes($G$ est fini), alors $[X]_a=[0]_a$, c'est-à-dire il exists $m \in \mathbb{Z}$ tel que $X=ma$. 

    De même, $[X]_b=[1]_b$, donc il existes $n \in \mathbb{Z}$ tel que $X=nb+1$. On a donc $1+nb=ma$, d'où $1=ma-nb=d(ma_1-nb_1)\in d\mathbb{Z}$. Mais c'est impossible car $d>1$. C'est donc l'absurde.
\end{itemize}
Finalement, on a donc 

\fbox{$G \times H$ est cyclique si et seulement si les
entiers $card(G)$ et $card(H)$ sont premiers entre eux}
\section{Exercice 2}
\subsection{}
Car $d = pgcd(k, n)$, il existe $(k_1,n_1) \in \mathbb{Z}^2$, tel que $k=dk_1$, $n=dn_1$, avec $k_1, n_1$ premiers entre eux. 

Pour $\overline{k} \in \mathbb{Z}/n\mathbb{Z}$
\begin{itemize}
    \item $\overline{k}^{n_1}=\overline{k}n_1=\overline{kn_1}=\overline{\frac{kn}{d}}=\overline{nk_1}=\overline{0}$, l'élément neutre de $\mathbb{Z}/n\mathbb{Z}$
    \item Soit $x\in \mathbb{Z}$ tel que $\overline{k}^{x}=\overline{0}$, donc $\overline{0}=\overline{k}x=\overline{kx}$, donc $n|kx$. Alors $\exists m \in \mathbb{Z}, mn=kx$, 
    c'est-à-dire $\exists m \in \mathbb{Z}, mn_1=xk_1$, c'est-à-dire $n_1|xk_1$. Car $k_1, n_1$ sont premiers entre eux, par le théorème de Gauss, on a $n_1|x$
\end{itemize}
Finalement, on a $\boxed{o(\overline{k})=n_1=\frac{n}{d}}$
\subsection{}
\begin{itemize}
    \item soit $x \in \langle \overline{k}\rangle$, alors il existe $a \in \mathbb{Z}$ tel que $x=\overline{k}^a$. Donc $x=\overline{ka}=\overline{dk_1a}=\overline{d}^{k_1a}\subset \langle \overline{d}\rangle$. 
    Donc $\langle \overline{k}\rangle \subset \langle \overline{d}\rangle$

    \item soit $x \in \langle \overline{d}\rangle$, alors il existe $b \in \mathbb{Z}$ tel que $x=\overline{d}^b$.
    Car $k_1, n_1$ sont premiers entre eux, par le théorème de Bezout, il existe $(u,v) \in \mathbb{Z}^2$ tel que $un_1+vk_1=1$
    \begin{itemize}
        \item analyse: Soit $x \in \langle \overline{k}\rangle$, alors il faut qu'il existe $c \in \mathbb{Z}$ tel que $x=\overline{db}=\overline{k}^c=\overline{kc}$
        , alors il existe $a \in \mathbb{Z}$ tel que $db+an=kc$, donc $b+an_1=ck_1$. Donc $b(un_1+vk_1)+an_1=ck_1$, on a alors $k_1|(bu+a)n_1$. On peut donc prendre $a=k_1-bu \in \mathbb{Z}$
        \item synthèse: on a $x=\overline{db}=\overline{db+(k_1-bu)n}=\overline{db+(k_1-bu)dn_1}=\overline{db+k_1dn_1-bdun_1}=\overline{db+k_1dn_1-bd(1-vk_1)}=\overline{k_1d(n_1+bv)}=\overline{k(n_1+bv)}=\overline{k}^{n_1+bv}\subset \langle \overline{k}\rangle$
        
        On a donc $\langle \overline{d}\rangle \subset \langle \overline{k}\rangle$
    \end{itemize}
\end{itemize}
Finalement, on en déduit $\boxed{\langle \overline{d}\rangle=\langle \overline{k}\rangle}$, ils donc engendrent le même sous-groupe de $\mathbb{Z}/n\mathbb{Z}$
\subsection{}
Par les résultat précédents, on considère l'ensemble $S=\bigcup_{e \in E}\{\langle \overline{e}\rangle\}$, avec $E=\{e|e \in [\![1,n]\!],e|n\}$, l'ensemble de diviseurs de $n$.
\begin{itemize}
    \item pout tout sous ensemble $\langle \overline{k}\rangle$ avec $k \in [\![1,n]\!]$, on note $d=pgcd(k,n)$, on a $d|n$, donc $d \in E$. Par les résultat précédents, on a $\langle \overline{k}\rangle=\langle \overline{d}\rangle \in S$
    
    l'ensemble de sous groupes de $\mathbb{Z}/n\mathbb{Z}$ est donc inclus dans $S$
    \item pour tous $s \in S$, il exise $e \in E \in [\![1,n]\!]$, tel que $s=\langle \overline{e}\rangle$. $s$ est bien un sous groupe de $\mathbb{Z}/n\mathbb{Z}$. 
    
    $S$ est donc inclus dans l'ensemble de sous groupes de $\mathbb{Z}/n\mathbb{Z}$

\end{itemize}
Finalement, l'ensemble de sous groupes de $\mathbb{Z}/n\mathbb{Z}$ est donnée par $\boxed{S=\bigcup_{e \in E}\{\langle \overline{e}\rangle\}}$

\end{document}